\section{Financiación} % (fold)
\label{sec:financiación}

\begin{mydef}[Concesión]
    La empresa se remunera mediante la explotación y dirección del servicio a cambio de asumir los costes de construcción. Es una forma de financiar un proyecto.
\end{mydef}

\begin{myrem}[Rescate concesión]
    El rescate se define como la recuperación por parte de la administración de una concesión. Cómo se paga este último ?
    \begin{itemize}
        \item Valor de mercado aunque es complejo ya que las concesiones no cotizan en bolsa.
        \item Valor contable
        \[
            \textsc{Inversión} - \textsc{Amortización} + \textsc{Beneficio Futuro Actualizado}
        \]
        \[
            \textsc{Beneficio Futuro Actualizado} = \sum\frac{B_{medio-anterior}}{(1+K)^n}
        \]
        \item Valor de flujos futuros
        \[
            \textsc{Patrimonio Neto} + \textsc{Flujos Futuros}
        \]
    \end{itemize}
\end{myrem}

\begin{mydef}[Credito participativo]
     Los préstamos participativos son préstamos en los que se estipula que el prestamista-financiador, además de la remuneración ordinaria a través de intereses, obtiene una remuneración dependiente de los beneficios obtenidos por el prestatario-financiado.
\end{mydef}

\begin{mydef}[Credito sindicado]
    Credito asumido por una agregación de prestamistas (entidad financiera…) para repartir el riesgo entre los mismos.
\end{mydef}

\begin{mydef}[Project finance]
    Project Finance, Financiación de Proyectos o Finanproyecto (traducción adaptada del vocablo inglés) es un mecanismo de financiación de inversiones de gran envergadura que se sustenta tanto en la capacidad del proyecto para generar flujos de caja que puedan atender la devolución de los préstamos como en contratos entre diversos participantes que aseguran la rentabilidad del proyecto. Es decir la empresa ``sponsor'' del proyecto ve su responsibilidad limitada a los activos mismo.
\end{mydef}

\begin{mydef}[Leasing/renting]
    Un alquiler con derecho a compra por el valor residual (renting normalmente no).
\end{mydef}

 \begin{mydef}[Factoring]
     El factoring es una alternativa de financiamiento que se orienta de preferencia a pequeñas y medianas empresas y consiste en un contrato mediante el cual una empresa traspasa el servicio de cobranza futura de los créditos y facturas existentes a su favor y a cambio obtiene de manera inmediata el dinero a que esas operaciones se refiere, aunque con un descuento.
 \end{mydef}

\begin{mydef}[Compañia de seguros]
    Re-invierten el dinero de los asegurados para generar beneficios mientras no declaran un siniestro.
\end{mydef}


\begin{mydef}[Concurso de creditores]
    Cuando una empresa no tiene la posibilidad de pagar sus deudas a corto plazo y no llega a ningún acuerdo con las sociedades de credito, ha de acudir a un concurso de creditores en el que un juez evalua su capacidad para hacer frente a las distintas deudas. En caso negativo se declara la empresa insolvente.
\end{mydef}

% section financiación (end)