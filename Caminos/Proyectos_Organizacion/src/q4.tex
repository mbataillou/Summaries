\section{Teoría} % (fold)
\label{sec:proyectos}

\begin{mydef}[Documentos proyectos cataluña]
	\begin{enumerate}
		\item Memoria (no contractual)
		\item Planos (contractuales)
		\item PPTP (contractual)
		\item Presupuesto (hay partes contractuales, véase el cuadro de precios 1 (generales) y 2 desglosados,  y no contractuales)
		\item [Anejo] Estudio Seguridad y salud
		\item [Anejo] Declaración de impacto ambiental
	\end{enumerate}
\end{mydef}

\begin{mydef}[Project management]
	Amplias áreas de actuación, gestión de: 
	\begin{itemize}
		\item Alcance 
		\item Tiempo
		\item Costes 
		\item Calidad
	\end{itemize}
	Control de: 
	\begin{itemize}
		\item RRHH
		\item Comunicaciones
		\item Riesgos
		\item Adquisiciones
	\end{itemize}
\end{mydef}

\begin{mydef}[Project manager]
	Realiza la \textbf{gestión} integral del proyecto (presencia temporal importante). No hace el proyecto sino que controla la calidad, el dinero y el tiempo. Es un gestor trabaja de forma más global.
\end{mydef}

\begin{mydef}[Director de obra]
	Interpreta el proyecto y lo adapta a la normativa vigente. Es una pieza más, tiene una misión ``concreta''.
\end{mydef}

\begin{mydef}[Project finance]
	``Asignación de riesgos entre proveedores de capital''. La garantia la tiene el propio proyecto. Los bancos se reparten el riesgo. Existen 2 tipos de riesgos:
	\begin{itemize}
		\item Riesgo en construcción: altos niveles de inversión (estudios previos, ejecución, expropiaciones).
		\item Riesgo associado:
		\begin{itemize}
			\item Demanda 
			\item Disponibilidad
		\end{itemize}
	\end{itemize}
\end{mydef}

\begin{mydef}[Administración concesión]
	Administración es reguladora. Establece leyes y criterios para controlar las concesiones.
\end{mydef}

\begin{mydef}[Responsabilitat Patrimonial de l’administració]
	Si el concesionario no puede dar el servicio la administración interviene, ya que puede tener un impacto directo sobre la sociedad (por ejemplo gestión del agua).
\end{mydef}

\begin{mydef}[Evaluación de impacto ambiental de proyectos]
	Timeline:
	\begin{itemize}
		\item Solicitud de inicio
		\item Estudio de impacto ambiental de proyectos
		\item Información pública de proyecto y EIA
		\item Análisis técnico del expediente de impacto
		\item Declaración de impacto ambiental
	\end{itemize}
	Criterios:
	\begin{itemize}
		\item Presencia
		\item Caracter generico
		\item Tipo de acción 
		\item Sinergia
		\item Temporalidad
		\item Duración 
		\item Reversibilidad
		\item Continuidad
		\item Periodicidad
	\end{itemize}
	Valoración de impactos:
	\begin{itemize}
		\item Compatible: rápida recuperación sin medidas correctoras.
		\item Moderado: recuperación con tiempo pero sin medidas (medidas simples).
		\item Severo: recuperación con tiempo más medidas complejas.
		\item Crítico: recuperación supera el umbral tolerable y no es recuperable independientemente de las medidas que aplicamos.
	\end{itemize}
\end{mydef}

% section proyectos (end)