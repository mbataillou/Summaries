\section{Selección de alternativas}\label{sec:selection_of_alternatives}

\begin{mydef}[Flujos de caja]
	Los flujos de caja representan el resultado de explotación sumado a las amortizaciones, es el flujo monetario de explotación.
	\[
		\text{Flujo de caja (FC)}= \text{Resultado de explotación (REX) + Amortización}
	\]
\end{mydef}

\begin{mydef}[Valor actualizado neto (VAN)]
	Es una estimación de la suma de los flujos de caja (actualizados) a lo largo de un periodo de interés (generalmente periodo de concesión).
	\[
		VAN = -I_0+ \sum_{i=0}^n \frac{FC_i}{(1+K)^i}
	\]
\end{mydef}

\begin{mydef}[Tasa de actualización]
	Rentabilidad máxima de una inversión sin riesgo, así pues mide actualización del valor que tendría nuestro dinero con total seguridad.
\end{mydef}

\begin{mydef}[Tasa interna de retorno]
	Es la tasa de actualización tal que el VAN es nulo.
	\[
		K \ | \quad VAN=0 \Leftrightarrow K \ | -I_0+ \sum_{i=0}^n \frac{FC_i}{(1+K)^i}=0
	\]
	
\end{mydef}

\begin{mydef}[Payback]
	Es el tiempo necesario para recuperar la inversión con una tasa de actualización dada.
	\[
		n \ | \quad VAN=0 \Leftrightarrow n \ | -I_0+ \sum_{i=0}^n \frac{FC_i}{(1+K)^i}=0
	\]
\end{mydef}

\begin{mydef}[Rentabilidad]
	Es el ratio de inversión (R), es decir cuanto se ha invertido (I) para obtener un beneficio (B) dado.
	\[
		R= \frac{B}{I}
	\]
\end{mydef}

\question{¿Diferencia entre un TIR de análisis de viabilidad y un TIR de análisis coste beneficio?}{Un TIR de análisis de viabilidad se realiza para como documento financiero que apoye la candidatura a un proyecto y cumpla con los requisitos de los accionistas. Generalmente trata de minimizar las demandas de estos para mejorar la oferta de cara a un concurso.}

\question{CCPSA estudio de rentabilidad. TIR 10 \%. Banco financia 60\% con un credito sin comisiones y con un interes anual de 6\%. CCPSA pide el crédito, cómo afectara a la rentabilidad de CCPSA ?}{Esta disminuira ya que se añadira una carga financiera al resultado ordinario. La cantidad invertida es la misma tan sólo se reparte el riesgo entre la empresa y el banco, de ahí los intereses (entre otras cosas)}

\question{Definir \textbf{bono}}{Es un instrumento de renta fija, el cual se suele comercializar mediante cupos que precisan precio nominal y interés, así cómo fecha de pago y metodología del mismo}

\question{¿ Preferencias reveladas?}{Un indicador objetivo (indiferente a las diversas opiniones), mediante el cúal se puede deducir o inferir un parámetro o variable.}

\question{Si el proyecto fuese una concesión, los costes del proyecto son costes de este análisis?}{Sí ya que para un análisis coste beneficio se considera que es una alocación de recursos realizada por la sociedad, independientemente del carácter público o privado de los fondos.}

\question{Si un concesionario paga un canon al ayuntamiento, se ha de tener en cuenta en el análisis?}{No, ya que es neutro. No representa ni un coste ni un beneficio.}

\question{¿ Se ha de tener en cuenta el incremento del PIB en al análisis coste beneficio?}{No, ya que constituye un impacto ecónomico el cual no forma parte del ACB. En nuestro abanico de análisis se tiene en cuenta los beneficios que impactan en el bienestar social y no ecónomico.}

\question{Y el sueldo de los trabajadores?}{De la misma forma que anteriormente, no.}

% subsubsection irr_internal_rate_of_return (end)
