\section{Cimentaciones profundas} % (fold)
\label{sec:cimentaciones_profundas}


\question{En el método Delft para el cálculo de la resistencia en punta de un pilote, los tramos utilizados incluyen material por encima del extremo inferior del pilote. ¿ Por qué ?}{
	Por los posibles mecanismos de rotura. Se incluye este material ya que la forma del bulbo de rotura, puede llegar a afectar el terreno por encima del extremo inferior del pilote.
}

\question{Para qué tipos de terreno utilizarías pilotes de barrena continua (CPI-8)? Indica cuáles son sus principales inconvenientes.}{
	Para suelos que se \textbf{desmoronan}. 

	\begin{minipage}[t]{0.5\textwidth}
	Ventajas:
	\begin{itemize}
		\item Barato y buen rendimiento
		\item No hace falta revestimiento
	\end{itemize}
	\end{minipage}%
	\begin{minipage}[t]{0.5\textwidth}
		Inconvenientes:
		\begin{itemize}
			\item Riesgo de corte del hormigonado.
		\end{itemize}
	\end{minipage}
}

\question{Define el concepto de eficiencia de un grupo de pilotes. ¿Qué valores suele tener en arenas?}{
	\[
		Eficiencia = \eta =  \frac{Q_i}{\sum_i Q_i} = \frac{\textsc{carga de falla del grupo de pilote}}{\textsc{suma de las cargas de falla individuales}}
	\]
	Si el espaciado es muy pequeño puede que el terreno tenga que aguantar y que la carga de un solo pilote $\rightarrow \nu<1$.
	En el caso de arrenas solemos tener $\nu>1$ así pues tomamos 1 como valor (el menor).
	\begin{myrem}[Encepado]
		Si existe encepado los pilotes actuan como un bloque, en caso de no existir tomaremos el peor caso entre bloque y conjunto de pilotes.
		Los asientos pueden ser problematicos con encepado ya que implica esfuerzos por movimiento de apoyo.
	\end{myrem}
}

\question{¿Qué es el fenómeno de la fricción negativa de un pilote? Indica 3 causas posibles del fenómeno.}{
	El suelo que rodea la parte superior del pilote asienta en relación al pilote, cambiando la dirección de las fuerzas de fricción (resistencia de fuste).
	Causas:
	\begin{itemize}
		\item Arcilla blanda sobre estrato duro (roca)
		\item Sobrecarga en el terreno
		\item Pilote hincado sobre el relleno (mal compactado)
	\end{itemize}
	Solución:
	\begin{itemize}
		\item Tener en cuenta la fricción negativa en el cálculo de $q_r$ y FS.
		\item Reducir la fricción alrededor del pilote
	\end{itemize}
}

% section cimentaciones_profundas (end)