\section{Empuje de tierras} % (fold)
\label{sec:empuje_de_tierras}

\subsection{Coeficiente de empuje} % (fold)
\label{sub:coeficiente_de_empuje}
\begin{mybox}{Coeficiente de empuje}
	El coeficiente de empuje $K$ relaciona las tensiones efectivas verticales con las horizontales:
	\[
		K_0 = \frac{\sigma_h^\prime}{\sigma_v^\prime}
	\]	
	\tcbsubtitle{\emph{Influencia de la historia del suelo}}
	\begin{itemize}
		\item Normalmente consolidados: $K_0$ se mantiene constante.
		\item Sobre consolidados: $K_0$ aumenta pasado un umbral.
	\end{itemize}

	\tcbsubtitle{\emph{Medición de $K_0$ en laboratorio}}

	\begin{itemize}
		\item Ensayos edométricos:
		\item Evaluación mediante medida de la succión
		\begin{myrem}[Húmedad]
			La succión aumenta mucho al secarse, hemos de evitar que la muestra pierda húmedad.
		\end{myrem}
	\end{itemize}
	
	\tcbsubtitle{\emph{Medición de $K_0$ en campo}}

	\begin{itemize}
		\item Presiómetro: es una prueba de referencia de forma teórica. Presenta grandes dispersiones al ser muy sensible a la alteración inducida.
		\item Dilatómetro: se mide la presión necesaria para mover la membrana.
		\item Métodos no destructivos (ensayo sísmico): se basa en la medición de la velocidad de las ondas de corte (S)
		\begin{myrem}[Deformaciones]
			Este método sobrevalora el módule de corte ya que este es dependiente de las deformaciones. A menor deformación mayor módulo.
		\end{myrem}
	\end{itemize}
\end{mybox}

\begin{mybox}{Estados de Rankine}
	Hipótesis:
	\begin{enumerate}
		\item Rotura plana: las lineas de contorno están en rotura 
		\item ``Suelo no cohesivo (c’=0) $\rightarrow$ friccional''
		\item ``No existen cargas aplicadas en superficie''
		\item Suelo homogéneo
		\item Suelo isotropo
		\item ``No hay agua $\rightarrow$ suelo seco''
	\end{enumerate}

	\tcbsubtitle{\emph{Empuje activo}}
	\[
		K_a = \frac{\sigma_{h_a}^\prime}{\sigma_{v_0}^\prime} = \frac{1-\sin(\phi^\prime)}{1+\sin(\phi^\prime)} = \tan^2\left(\frac{\pi}{4} - \frac{\phi^\prime}{2}\right) <1
	\]	
	\tcbsubtitle{\emph{Empuje pasivo}}
	\[
		K_p = \frac{\sigma_{h_p}^\prime}{\sigma_{v_0}^\prime} = \frac{1}{K_a} = \frac{1+\sin(\phi^\prime)}{1-\sin(\phi^\prime)} = \tan^2\left(\frac{\pi}{4} + \frac{\phi^\prime}{2}\right) >1
	\]	
	\tcbsubtitle{\emph{Resumen de casos}}

	\begin{table}[H]
	\centering
	\begin{tabular}{ccc}
	\toprule
	Suelo & Activo & Pasivo \\
	\midrule
	Drenado - Cohesivo & $\sigma_{h_a}^\prime = K_a \sigma_{v_0}$ & $\sigma_{h_p}^\prime = K_p \sigma_{v_0}$ \\
	Drenado - No cohesivo & $\sigma_{h_a}^\prime = K_a \sigma_{v_0} - 2c^\prime \sqrt{K_a}$ &  $\sigma_{h_p}^\prime = K_p \sigma_{v_0}+ 2c^\prime \sqrt{K_p}$  \\
	No drenado & $\sigma_{h_a} = \sigma_v - 2C_u$& $\sigma_{h_p} = \sigma_v + 2C_u$ \\
	\bottomrule
	\end{tabular}
	\end{table}
\end{mybox}

\begin{mybox}{Método de Coulomb}
	Hipótesis:
	\begin{enumerate}
		\item Todo el suelo esta en rotura
		\item La superficie de rotura es plana
		\item Rozamiento suelo/muro es nulo $\Rightarrow$ Tensiones principales
	\end{enumerate}
	\begin{myrem}
		No se recomienda su uso para el cálculo del empuje pasivo ya que nos deja del lado de la inseguridad al sobrevalorarlo.
	\end{myrem}

	\tcbsubtitle{\emph{Solución básica}}

	\begin{gather}\label{eq:2.1}
      \frac{E}{\sin(\rho - \phi^\prime )}= \frac{W}{\sin(\alpha - \delta + \rho - \phi^\prime)} \\
      \shortintertext{siendo}
      \begin{aligned}
        &\alpha : \emph{ ángulo de inclinacón del muro}
        & \delta \in [0, \phi] \emph{ ángulo de rozamiento tierra muro}
      \end{aligned}\notag
    \end{gather}
    \begin{myrem}[Dirección del movimiento]
    	La dirección del movimiento viene marcada por el sentido de las fuerzas de rozamiento
    	\[
    		\tau = \sigma_n^\prime \tan(\delta)
    	\]
    \end{myrem}
\end{mybox}

% subsection coeficiente_de_empuje (end)


% section empuje_de_tierras (end)