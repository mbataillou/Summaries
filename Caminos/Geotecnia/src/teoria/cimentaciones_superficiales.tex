\section{Cimentaciones superficiales} % (fold)
\label{sec:cimentaciones_superficiales}

\begin{mydef}[Modos de rotura]
	Los distintos métodos son:
	\begin{enumerate}
		\item Punzonamiento
		\item Rotura global
		\item Rotura local
	\end{enumerate}
\end{mydef}

\begin{mybox}{Capacidad portante}
Es la máxima presión que puede sufrir el terreno bajo la cimentación
\tcbsubtitle{\emph{Caso drenado}}
	\begin{equation}
		q_r = cN_c + qN_q + \frac{1}{2}\gamma B N_r
		\label{eq:capacidad_portante}
	\end{equation}
\tcbsubtitle{\emph{Caso no drenado}}
	Esta situación se da a corto plazo y es la más critica.
	\begin{equation}
		q_r = C_uN_r + q
	\end{equation}
\tcbsubtitle{\emph{Caso submergido}}
	En un terreno submergido se producen tensiones efectivas así pues la carga de rotura también aparece en efectivas
	\begin{equation}
		\begin{cases}
			q_r^\prime = c^\prime N_c + q^\prime N_q + \frac{1}{2}\gamma^\prime B N_r, &\emph{tensiones efectivas}\\
			q_r = q_r^\prime + (P_w)_b, &\emph{tensiones totales}
		\end{cases}
	\end{equation}
	\begin{myrem}[Equivalencia]
		La expresión es equivalente a la anterior
	\end{myrem}
\end{mybox}

\begin{mybox}{Asientos}
Los asientos se descomponen en tres fases véase \ref{eq:asientos_totales}
\begin{equation}
	s = s_i + s_c + s_s
	\label{eq:asientos_totales}
\end{equation}
\tcbsubtitle{\emph{Elasticidad}}
La teoría elástica nos permite obtener la expresión de los asientos instantaneos
\begin{equation}
	s_i = \frac{qB(1-\nu^2)}{E}
\end{equation}
Aunque al ser instantaneos suelen asociarse a condiciones no drenadas. Luego 
\begin{equation}
	s_i = f(E_u , \nu_u = 0.5)
\end{equation}
Dónde
\[
	E_u = \frac{3E^{’}}{2(1+\nu^{’})}
\]
\begin{myrem}[Asiento total]
	Si omitimos los asientos secundarios y trabajamos en tensiones efectivas obtenemos los asientos totales
\end{myrem}
\tcbsubtitle{\emph{Método edométrico}}
	Hipótesis:
	\begin{enumerate}
		\item Se puede calcular el incremento de tensión vertical por la teoría elástica
		\item El incremento de presión intersticial es igual al incremento de tensión vertical
		\[
			\Delta u = \Delta \sigma_v
		\]
		\item Tensiones totales invariantes a lo largo de la consolidación
	\end{enumerate}
	Luego obtenemos:
	\[
		s_c = \int m \Delta \sigma_v = \int \frac{C_c}{1+e}\log\left(\frac{\sigma_{r_0}^{’}+\Delta\sigma^{’}}{\sigma_{r_0}^{’}}\right)		
	\]
\tcbsubtitle{\emph{Método de Skempton-Bjerrum}}
	Este método trata de corregir los efectos causados por la segundo hipótesis del método edométrico.
	\begin{equation}
		\Delta u = \Delta \sigma_h + A [\Delta \sigma_v + \Delta \sigma_h]
	\end{equation}
	Donde A depende del tipo de suelo.
	Luego:
	\begin{equation}
		S_{SB} =\mu S_{ed} = [A + \alpha (1-A)]S_{ed}
	\end{equation}

\end{mybox}

% section cimentaciones_superficiales (end)
