\section{Reconocimiento del terreno} % (fold)
\label{sec:reconocimiento_del_terreno}

\subsection{Tipos de reconocimiento} % (fold)
\label{sub:tipos_de_reconocimiento}

% subsection tipos_de_reconocimiento (end)
\begin{mybox}{Tipos de reconocimiento}
	El reconocimiento se constituye de 3 partes:
	\begin{enumerate}
		\item Recogida de documentación
		\item Reconocimiento previo en superficie
		\item Reconocimiento profundidad
	\end{enumerate}

	\tcbsubtitle{\emph{Catas}}
	\begin{ldef}[Catas]
		Excavar una zanja y observar
	\end{ldef}
	Problemas:
	\begin{itemize}
		\item Muy peligrosas
		\item Limitadas en profundidad
	\end{itemize}

	\tcbsubtitle{\emph{Galerías de reconocimiento}}
	\begin{ldef}[Galerías de reconocimiento]
		Hacer un túnel de pequeño diámetro para reconocer
	\end{ldef}

	\tcbsubtitle{\emph{Sondeos}}
	\begin{ldef}[Sondeos]
		Agujero en el terreno sin limitaciones de profundidad
	\end{ldef}
	\begin{itemize}
		\item A rotación con corona: empule + rotación con duentes para penetrar. Sacamos el testigo mediante maquinaria.
		\begin{itemize}
			\item Coronas 
			\begin{equation}
				\begin{cases}
					\emph{Tubo sencillo}: &\text{ alteradas por el agua}\\
					\emph{Tubo doble}: & \text{ no alteradas por agua, ni sometidas a rotación}
				\end{cases}\notag
			\end{equation}
			\item Lodos bentoníticos: bentonita (4-7\%), arcilla de alta plasticidad, que evita que el terreno se derrumbe. 
			\begin{myrem}[Lodos]
				Pueden alterar mucho el resultado (p.e permeabilidad), en su lugar se puede usar tuberia de revestimiento
			\end{myrem}
		\end{itemize}
		\item Off-shore: sondeos en el mar. Avance lento
		\item A sección completa: no necesita sacar el testigo tramo a tramo sino que nos permita sacar la sólo la sección que nos interesa.
		\item ``Wire line'': tecnología que permite sacar el testigo sin sacar la maniobra completa.
	\end{itemize}
\end{mybox}

\subsection{Muestras} % (fold)
\label{sub:muestras}

% subsection muestras (end)

\begin{mybox}{Muestras}
	Se clasifican en dos tipos:
	\begin{itemize}
		\item No representativas: faltan componentes importantes del terreno (p.e lavado de finos).
		\item Representativas:
			\begin{equation}
					\begin{cases}
						\emph{Alteradas}\\
						\emph{No alteradas}
					\end{cases}\notag
			\end{equation}
	\end{itemize}
	\begin{myrem}[Humedad]
		Es importante impedir que la muestra pierda humedad.
	\end{myrem}
	\tcbsubtitle{\emph{Bloque}}
		\begin{itemize}
			\item Las menos alteradas
			\item Limitaciones de profundidad y coste
			\item Perdida de humedad
		\end{itemize}
	\tcbsubtitle{\emph{Tubos toma-muestras}}
		\begin{itemize}
			\item Pared delgada (Shelby): terrenos blandos, se suelen hincar a presión.
			\item Pared gruesa: terrenos más cohesivos, se suelen hincar a golpes.
		\end{itemize}
	\tcbsubtitle{\emph{Tubos toma-muestras de piston fijo}}
		\begin{itemize}
			\item Terrenos blandos, se suelen hincar a presión pero el piston evita la compresión de la muestra $\Rightarrow$ mayor calidad.
		\end{itemize}
	\tcbsubtitle{\emph{Tubos sacatestigos}}
		\begin{itemize}
			\item Materiales arcillosos sensibles al contacto con el agua.
		\end{itemize}
\end{mybox}

\subsection{Ensayos ``in-situ''auie} % (fold)
\label{sub:ensayos_}

% subsection ensayos_ (end)

\begin{mybox}{Ensayos ``in-situ''}
	\tcbsubtitle{\emph{SPT}}
	\begin{ldef}[SPT]
		Medición del número de golpes necesarios para hincar los 30cm intermedios de una cuchara con peso de 63,5Kg desde una altura de 760mm
	\end{ldef}
	\begin{ldef}[Rechazo]
		Si al cabo de 100 golpes no ha penetrado en el terreno.
	\end{ldef}
	\begin{myrem}[Correcciones]
		Se realizan dos correcciones principales
		\begin{equation}
			\begin{cases}
				N_1 = C_N N & \emph{Por profundidad}\\
				N_{60} = \frac{N_{SPT}E_{SPT}}{E_{60}} & \emph{Por energía}
			\end{cases}
		\end{equation}
	\end{myrem}

	\tcbsubtitle{\emph{CPT}}
	\begin{ldef}[CPT]
		Resistencia del terreno a la introducción de un cono normalizado
	\end{ldef}
	Parámetros obtenidos:
	\begin{equation}
		\begin{cases}
			q_c = \frac{F}{A_{cono}}, & \emph{ resistencia a penetración}\\
			f = \frac{F_{LM}}{A_{manguito}} & \emph{ tensión lateral}
		\end{cases}
	\end{equation}
	\begin{minipage}[t]{0.5\textwidth}
	Ventajas:
	\begin{enumerate}
		\item Muy repetible
		\item Incado a presión $\Rightarrow$ mayor calidad
		\item Registro continuo ($v= 2 m/s$)
	\end{enumerate}
	\end{minipage}%
	\begin{minipage}[t]{0.5\textwidth}
	Inconvenientes:
		\begin{itemize}
			\item Necesidad de una reacción importante en el terreno.
			\item No se ve el terreno atravesado
		\end{itemize}
	\end{minipage}

	\tcbsubtitle{\emph{$CPT_u$}}
	
	\begin{ldef}[CPT]
		Resistencia del terreno a la introducción de un cono normalizado considerando las presiones de agua
	\end{ldef}
	Parámetro obtenido
	\begin{gather}\label{eq:2.1}
      q_t = q_c + (1-a)u_2 \\
      \shortintertext{siendo}
      \begin{aligned}
        &a = \frac{A_{bastago}}{A_{cono}}
      \end{aligned}\notag
    \end{gather}

    \begin{myrem}[Coeficiente de consolidación ($C_h$)]
    	Se puede calcular mediante un ensayo de disipación. Medimos $t_{50}$, el tiempo que tarda en disiparse el 50\% de la presión intersticial
    \end{myrem}

    \tcbsubtitle{\emph{Penetración dinámica}}
    \begin{ldef}[Penetración dinámica]
    	Hincar una varilla en el terreno con una puntaza.
    \end{ldef}
    Parámetro obtenido:
    \[
    	q_d = \frac{R_d}{A_d}
    \]
    \begin{minipage}[t]{0.5\textwidth}
	Ventajas:
	\begin{enumerate}
		\item Rápido
		\item Barato
		\item Poca maquinaria
	\end{enumerate}
	\end{minipage}%
	\begin{minipage}[t]{0.5\textwidth}
	Inconvenientes:
		\begin{itemize}
			\item Profundidad limitada
			\item No se ve el terreno atravesado
		\end{itemize}
	\end{minipage}

	\tcbsubtitle{\emph{``Vane-test'' (molinete)}}
	Permite la obtención de $C_u$ en arcillas medias y blandas

	\begin{myrem}[Correcciones]
		Se realizan correcciones por 
		\begin{equation}
			\begin{cases}
				\emph{Velocidad de carga}\\
				\emph{Anisotropía}
			\end{cases}
		\end{equation}
		\begin{gather}\label{eq:2.1}
	     (C_u)_{campo} = \mu (C_u)_{Vane-Test} \\
	      \shortintertext{siendo}
	      \begin{aligned}
	        &\mu  :\emph{ parámetro corrector}\\
	        & \frac{d \mu}{dIp}
	      \end{aligned}\notag
	    \end{gather}
	\end{myrem}

	\tcbsubtitle{\emph{Presiómetro}}

	\begin{itemize}
		\item Menard: aplicamos una presión a una membrana hinchable y medimos desplazamientos producidos dentro de un sondeo
		\item Autoperforador: minimiza la alteración del terreno al medir a la vez que perfora.
		\begin{equation}
			\begin{cases}
				\emph{Para terreno blando}\\
				\emph{Más caro}\\
				\emph{Más díficil de manejar}
			\end{cases}
		\end{equation}
	\end{itemize}

	\tcbsubtitle{\emph{Ensayos sísmicos}}

	Miden la velocidad de prodagación de los tipos de onda:
	\begin{equation}
			\begin{cases}
				\emph{Ondas S: corte} \rightarrow \emph{sólo se desplazan por el suelo}\\
				\emph{Ondas P: compresión} \rightarrow \emph{desplazan por agua}
			\end{cases}
	\end{equation}
	Tipos de ensayo:
	\begin{itemize}
		\item Cross-hole: emitimos ondas $P$ y $S$ entre un emisor y un receptor, a lo largo de una distancia $d$ conocida. Se necesitán dos sondeos.
		\item Up-hole/down-hole: mismo procediminto sólo con un sondeo (menos calidad)
		\item Cono sísmico: medida de velocidades a medida que penetra el cono
		\item Ondas Ray-Leigh: se emiten ondas de superficie mediante martillo vibrador continuo. No se necesita sondeo pero los resultados son más díficiles de interpretar
	\end{itemize}

	\tcbsubtitle{\emph{Ensayos hidráulicos}}

	\begin{itemize}
		\item Lefranc: permite calcular la permeabilidad en un sondeo
		\[
			Q = FKH
		\]
		\begin{equation}
			\begin{cases}
				\emph{Carga constante} \rightarrow \emph{materiales con alta permeabilidad}\\
				\emph{Carga variable} \rightarrow \emph{materiales con baja permeabilidad}
			\end{cases}
	\end{equation}
		\item Lugeon: permite calcular la permeabilidad en rocas fracturadas. Se mide el caudal capaz de infiltrarse en rocas a presión dada.
	\end{itemize}
\end{mybox}

\begin{mybox}[sidebyside,sidebyside align =top]{``In-situ'' VS Laboratorio}
	\tcbsubtitle{\emph{Ensayos ``in-situ''}}
	\begin{itemize}
		\item Se ensaya más volumen de suelo
		\item Información contínua
		\item Suelo estado natural (no hay alteración)
		\item Más rapido que en la realidad
		\item Uso de correlaciones empíricas…
	\end{itemize}
	\tcblower
	\tcbsubtitle{\emph{Ensayos laboratorio}}
	\begin{itemize}
		\item Uniformidad tensiones
		\item Suelo bien identificado
		\item Alteración de muestras
	\end{itemize}
\end{mybox}
% section reconocimiento_del_terreno (end)
