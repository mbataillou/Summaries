
\section{Financement de l'économie et rôle de la monnaie} % (fold)

\label{prt:financement_de_l_economie_et_role_de_la_monnaie}

À quoit sert la monnaie ? Plusieurs questions se posent, ce chapitre a pour but d'apporter des éléments de réponse. 

La première question que l'on se pose est : qu'est-ce qu'un besoin de financement ? 
Un besoin de financement représente le fait qu'un agent ne puisse financer ses besoins avec ses ressources. Il recourt alors à un \emph{système de crédit} ou un 
\emph{système de financement} (distinction faite par Hicks en 1974).

On note que les ménages et entreprises possèdent une capacité de financement mais elle peut s'avérer insuffisante. Un point à analyser est le fait qu'à partir 
des années 80, l'état adopte un position favorable à l'endettement (actuellement l'endettement représente $90\%$ du PIB).

À l'échelle microéconomique,  ce sont les banques qui financent les différents agents. On favorise ainsi la circulation de la monnaie, de par une promesse de
remboursement futur. Cependant on encourt le risque d'\emph{inflation}. Ce risque s'équilibre par l'augmentation de la demande globale et donc de la croissance 
(augmentation revenus..), éléments qui permettent de rembourser les différents emprunts. Mais si les revenus stagnent, on peut entrer dans une impossibilité de remboursement qui entraine un nouvel emprunt, en arrivant finalement à un surendettement.

Et les marchés financiers ? Ils représentent la deuxième source de financement. Ils permettent d'obtenir différents capitaux, sans passer par les banques. On 
est cependant à la merci des différents risques associés aux méthodes financières (spéculation..). Ces marchés peuvant financer des projets sans aucune finalité d'investissement productif... On crée ainsi des \emph{bulles spéculatives}.

La crise des \emph{subprimes} est un exemple de bulle spéculative. Elle concernait les ménages americains insolvables qui s'étaient endettés pour acheter des
biens immobiliers (avec des crédits dits à prêt hypothécaires, basés sur la valeur du logement). La chute du prix de l'immobilier et la hausse du taux d'intérêt a 
provoqué la crise. Mais comment en est-on arrivé là ? On créa des titres à partir de dettes hypothécaires, les banques créaient un prêt hypothécaire, puis réunissaient 
certaines hypothèques en créant des titres "attractifs" qui entraient sur les marchés. Les banques ne supportaient aucun risque, et réalisaient des prêts 
insolvables. Cependant, l'insolvabilité d'une grande partie des ménages a provoqué la chute de beaucoup de banques (il est important de comprendre que les 
banques sont aussi des acteurs financiers..). Cet effet provoqua un endurcissement des prêts qui a ralenti la création de valeurs et la croissance de 
l'économie. Plusieurs plans de rachat par l'État des créances douteuses se sont mis en place. Un image importante de cette crise est la globalité de 
l'économie, la crise americaine a fait trembler l'économie mondiale. 

Quelles mesures aujourd'hui ? Plusieurs institutions (Conseil européen du risque financier, Plan de refonte de la régulation de la finance américaine) ont été
 instaurées pour contrôler les actions des marchés financiers. De plus au niveau international le FMI évolue pour répondre aux enjeux de la crise. 
 
Pour analyser les enjeux économiques actuels il nous faut présenter certaines notions.

\begin{tcolorbox}[title=Les agrégats monétaires]
	Ce sont des regroupements d'ensembles homogènes d'actifs monétaires ou non. On les classe ensuite par ordre de liquidité décroissante. 
	\begin{itemize}[label=\ding{69}]
		\item M1: monnaie fiduciaire (billets) et les dépôts à vue.
		\item M2: somme de M1 et des depôts moins liquides, par exemple ceux qui peuvent être liquides après un certain terme (Livret A..)
		\item M3: M1+ M2 auxquels on ajoute les exigibilités négociables des institutions financières.
	\end{itemize}
	
\end{tcolorbox}

\subsection{Le passage de l'économie d'endettement à celle de marché financier} % (fold)
\label{sec:le_passage_de_l_economie_d_endettement_a_celle_de_marche_financier}

Les années de faible croissance et forte inflation (70), entrainent un besoin d'endettement de l'État. Mais le ralentissement de la croissance dû notamment à 
l'alourdissement des charges financières, entraine un surendettement de ce dernier.

\subsubsection{La lutte contre l'inflation} % (fold)
\label{sub:la_lutte_contre_l_inflation}

Pour réduire l'inflation on augmente le taux d'intérêt à fin de réduire l'octroi de crédits. Il faut donc que l'État et les entreprises se procurent des 
ressources sur le marché des capitaux. L'épargne sur ces marchés étant relativement faible, c'est l'État qui l'obtient au détriment des entreprises (on parle
d'\emph{effet d'éviction}).

On distingue 2 types de marchés de capitaux : 
\begin{itemize}[label=\ding{69}]
	\item Marché monétaire : il comprend le marché interbancaire (échange de capitaux à court terme) et le marché de titres et de créances.
	\item Marché financier : capitaux à long terme, on distingue 2 types de titres. Les actions et les obligations. Les variations d'intérêts sont très influentes sur ces marchés (pour y remedier les agents ont créé des marchés à terme régularisés par la MONEP ou la MATIF).
\end{itemize}

% subsubsection la_lutte_contre_l_inflation (end)

\subsubsection{Vers l'économie de marché des capitaux} % (fold)
\label{sub:vers_l_economie_de_marche_des_capitaux}

Les années 90 voient une explosion des transactions sur le marché financier, ce qui restreint le type d'entreprise se finançant chez une banque aux entreprises
les plus risquées. Les banques se voient-elles affectées ? En faits les banques sont des acteurs directs de l'épargne, elles possèdent l'information (asymétrie)
. Les États et entreprises demandent du capital sur les marchés financiers qui contactent les banques pour obtenir des informations sur les épargnants, elles 
sont donc des acteurs clés de cette explosion du marché financier.

% subsubsection vers_l_economie_de_marche_des_capitaux (end)

\subsubsection{La globalisation financière et ses risques} % (fold)
\label{sub:la_globalisation_financiere_et_ses_risques}

On définit la globalisation à l'aide de trois phénomènes concomitants : 
\begin{itemize}[label=\ding{69}]
	\item La déréglementation, autorisation des mouvements de capitaux entre pays.
	\item La désintermédiation, entraîne nombre de participants plus élevés sur les marchés de capitaux.
	\item Décloisonnement des marchés de capitaux, toutes les formes de financement rencontrent tous les besoins de financement (plus de distinction entre
	marché à court et à long terme) 
\end{itemize}

Un intérêt clair lié à la globalisation est le fait que les pays en excès de capital pourront subsister aux besoins des pays nécessitant un prêt. Par 
exemple les USA ont comblé leur déficit de transactions courantes (dû aux importations chinoises) avec un apport de capital (lui aussi chinois).

On voit aussi apparaître des fusions internationales d'entreprises, cependant une nouvelle gouvernance prend place, le désir de transparence et de rentabilité entraîne des licenciements et des fortes prises de risques dans les entreprises entrainant leur faillite.

% subsubsection la_globalisation_financiere_et_ses_risques (end)


% subsection le_passage_de_l_economie_d_endettement_a_celle_de_marche_financier (end)

\subsection{Le rôle de la monnaie} % (fold)
\label{sec:le_role_de_la_monnaie}

\subsubsection{Les fonctions de la monnaie} % (fold)
\label{sub:les_fonctions_de_la_monnaie}

On discerne 3 grandes fonctions de la monnaie : 

\begin{itemize}[label=\ding{69}]
	\item Intermédiaire des échanges, on peut toujours échanger de la monnaie contre un bien. Elle va faire office d'intermédiaire à l'achat d'un bien futur.
	\item Unité de compte,on peut exprimer la valeur d'un bien en valeur absolue.
	\item Réserve de valeur, elle permet un stockage dans le temps du pouvoir d'achat.
\end{itemize}

% subsubsection les_fonctions_de_la_monnaie (end)

\subsubsection{La demande d'encaisse monétaire ou demande de monnaie} % (fold)
\label{sub:la_demande_d_encaisse_monetaire_ou_demande_de_monnaie}
Dans la pensée classique une augmentation du volume de la monnaie entraine une augmentation du prix des transactions (en supposant que leurs volumes et vitesses restent constants). $Mv=pT$

La critique keynésienne, remet en cause la neutralité de la monnaie. Il stipule qu'il existe des préferences dans le type des moyens de paiement (liquidité) et que l'on peut désirer la monnaie pour stocker de la valeur.

La valeur de la demande peut varier de par cette analyse. En effet les agents peuvent préferer des placement à taux fixes. Ces placements varient en valeur
en fonction du taux d'intérêt (acheter avec taux fort \& revendre à taux faible). Ce qui remet en cause l'analyse classique qui stipule une indépendance entre 
la sphère réelle et la sphère monétaire.

\begin{tcolorbox}[title=La demande de monnaie dans une économie keynésienne simplifiée]
	Rôles de la monnaie : 
	\begin{itemize}
		\item Moyen de paiement lors des échanges économiques. On voit donc apparaître la préférence pour un argent liquide qui permet de réaliser ces échanges.
		\item Support de l'épargne, actif monétaire. Cet actif permet de transférer un pouvoir d'achat dans le temps.
	\end{itemize}

\emph{La spéculation.} Un spéculateur essaie de créer de la plus-value en capital. Acheter un actif quand son prix est bas et le revendre quand il est haut.

\emph{Arbitrage}. Dans l'analyse keynésienne il existe deux supports de réserve de valeur: les actifs monétaires et les financiers. Elle privilégie les 
financiers car ils donnent une relation simpe entre le taux d'intérêt et le prix des obligations. 

On observe que si le taux d'intérêt augmente, le prix d'un titre baisse et la demande de titres augmente (demande de monnaie baisse), et l'inverse si le taux
d'intérêt diminue.
	
\end{tcolorbox}

Une autre opinion se forge à partir des années 70, Friedman, stipule que la demande de monnaie dépend des revenus permanents (somme des revenus attendus du 
patrimoine humain et matériel). Le revenu permanent étant moins sensible aux fluctuations, la demande de monnaie est stable par rapport à ce dernier. 
Donc si la masse monétaire augmente les agents vont modifier leur portefeuille et ainsi faire monter les prix. Ce qui rejoint le vision classique, et 
s'oppose à la vision keynesienne (qui accorde l'importance au revenu courant).

% subsubsection la_demande_d_encaisse_monetaire_ou_demande_de_monnaie (end)



% subsection le_role_de_la_monnaie (end)

\subsection{Le rôle de la banque centrale} % (fold)
\label{sec:le_role_de_la_banque_centrale}

Dans le système bancaire, une seule banque peut émettre des billets, créer de la monnaie centrale. C'est, comme son nom l'indique, la banque centrale. 
Les banques commerciales peuvent réaliser de la création monétaire en octroyant des crédits, cependant elle se doivent de garder une réserve de monnaie
centrale. Elles vont donc se refinancer (en émettant des obligations sur les différents marchés). Mais quelle est la relation entre création monétaire
et monnaie centrale ? 
\begin{itemize}
	\item La première stipule que la création monétaire à un crédit est directement liée à la monnaie centrale. En stipulant que l'augmentation de monnaie centrale induit une augmentation de la création monnétaire \emph{(multiplicateur de crédit)}.
	\item La deuxième donne le "pouvoir" aux banques commerciales. Elles détiennent l'initiative quant aux crédits, et elles en déterminent le montant
	en fonction des besoins de l'économie (\emph{diviseur de crédit}) 
\end{itemize}

La banque centrale intervient aussi sur le taux d'intérêt, elle le fait varier en fonction de la quantité de monnaie à injecter. Elle vise le taux plancher si c'est elle qui décide d'injecter l'argent et le taux plafond si c'est à la demande d'une banque comerciale. 

La dernière forme d'intervention de la banque centrale est la politique d'\emph{open market}, qui définit l'achat ou vente de titres par la banque centrale.

% subsection le_role_de_la_banque_centrale (end)


% part financement_de_l_economie_et_role_de_la_monnaie (end)

