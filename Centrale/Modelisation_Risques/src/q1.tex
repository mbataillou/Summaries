Les fonctions permettant de calculer les différents influences et valeurs de cette section sont Listing~\ref{listing:10}.

Nous présentons dans les Figure~\ref{fig:call_s}, Figure~\ref{fig:call_sigma}, et Figure~\ref{fig:call_t}, l'évolution du prix d'un $CALL$ évalué selon la formule de Black et Scholes.

\begin{equation} C_{t}=S_{t}N(d_{1})-Ke^{-r(T-t)}N(d_{2})\end{equation}
où

\begin{align*}
N(d) &= \frac{1}{\sqrt{2\pi}}\int_{-\infty}^{d} {e^{\frac{-q^{2}}{2}} dq}\\\ 
d_1 &=\frac{1}{\sigma\sqrt{T-t}}(ln(\frac{S}{K})+(r+\frac{\sigma^{2}}{2})(T-t))\\
d_2 &=d_1-\sigma\sqrt{T-t}
\end{align*}

 On pourra se référer au document \textsc{black$\_$scholes.m} pour analyser l'implémentation de la formule sous \textsc{Matlab}. Pour simuler ce prix, on va prendre les paramètres suivants: le prix du sous-jacent $S_0=75$, le Strike $K=75$, l'échéance $T=1$, la volatilité $\sigma = 0.17$ et le taux d-actialisation $r=0.01$.
Avec ces paramètres, on obtient le prix d'un call européen $P_{B-S}=5.437$ en applicant le formule de Black et Scholes. 

\vfill

\begin{figure}[H]
\centering
% This file was created by matlab2tikz.
%
%The latest updates can be retrieved from
%  http://www.mathworks.com/matlabcentral/fileexchange/22022-matlab2tikz-matlab2tikz
%where you can also make suggestions and rate matlab2tikz.
%
\begin{tikzpicture}

\begin{axis}[%
width=0.6\linewidth,
height=0.6\linewidth,
at={(1.011in,0.642in)},
scale only axis,
xmin=0,
xmax=150,
xlabel style={font=\color{white!15!black}},
xlabel={$\text{S}_\text{0}$},
ymin=0,
ymax=80,
ylabel style={font=\color{white!15!black}},
ylabel={CALL},
axis background/.style={fill=white}
]
\addplot [color=black, line width=0.8pt, forget plot]
  table[row sep=crcr]{%
1	3.48452696036907e-143\\
1.5	5.32600750164986e-118\\
2	1.26513108472006e-101\\
2.5	8.94026972934864e-90\\
3	1.20260833302136e-80\\
3.5	2.57512625343218e-73\\
4	2.97861317801596e-67\\
4.5	3.99205620062664e-62\\
5	1.02853963445346e-57\\
5.5	7.24220129373355e-54\\
6	1.79588886128879e-50\\
6.5	1.89340652279537e-47\\
7	9.79573527802002e-45\\
7.5	2.78015911958792e-42\\
8	4.7277447299704e-40\\
8.5	5.17109753032313e-38\\
9	3.85425995649898e-36\\
9.5	2.05309577204604e-34\\
10	8.13237705632921e-33\\
10.5	2.4766110303593e-31\\
11	5.96517254319488e-30\\
11.5	1.16418294485456e-28\\
12	1.87974057884603e-27\\
12.5	2.5567560764831e-26\\
13	2.9759379630785e-25\\
13.5	3.00527432039802e-24\\
14	2.665246655631e-23\\
14.5	2.09818058683012e-22\\
15	1.48026429914428e-21\\
15.5	9.43882198771932e-21\\
16	5.48128912012728e-20\\
16.5	2.91878555260458e-19\\
17	1.43400681807623e-18\\
17.5	6.53654033555242e-18\\
18	2.7783353102926e-17\\
18.5	1.1062525330171e-16\\
19	4.14355545212528e-16\\
19.5	1.46555034040817e-15\\
20	4.91198049836248e-15\\
20.5	1.56507593585234e-14\\
21	4.75466130711325e-14\\
21.5	1.38099123228189e-13\\
22	3.8445049322425e-13\\
22.5	1.02819882878888e-12\\
23	2.64749873185482e-12\\
23.5	6.57633366594322e-12\\
24	1.57880162878873e-11\\
24.5	3.66958864119078e-11\\
25	8.27090103750631e-11\\
25.5	1.8104513906841e-10\\
26	3.85416095545765e-10\\
26.5	7.99011370906182e-10\\
27	1.61507033292347e-09\\
27.5	3.18674573336951e-09\\
28	6.14458397481216e-09\\
28.5	1.15896434868453e-08\\
29	2.14041291435038e-08\\
29.5	3.87407650857148e-08\\
30	6.8778525522472e-08\\
30.5	1.1986757138566e-07\\
31	2.05232239122842e-07\\
31.5	3.45459710618581e-07\\
32	5.72075528731744e-07\\
32.5	9.32597178113048e-07\\
33	1.49756490395498e-06\\
33.5	2.37017557071811e-06\\
34	3.69929062465313e-06\\
34.5	5.6967520313188e-06\\
35	8.66011808261261e-06\\
35.5	1.30021204741038e-05\\
36	1.92883398043171e-05\\
36.5	2.82847918731098e-05\\
37	4.10173036194069e-05\\
37.5	5.88447256981322e-05\\
38	8.35481679745541e-05\\
38.5	0.000117438543361212\\
39	0.000163484752970889\\
39.5	0.000225464830347128\\
40	0.00030814227427732\\
40.5	0.000417469629572665\\
41	0.000560821116434046\\
41.5	0.000747255757396752\\
42	0.000987812005157143\\
42.5	0.0012958343369803\\
43	0.0016873316575745\\
43.5	0.00218136665162482\\
44	0.00280047446244665\\
44.5	0.00357110826055682\\
45	0.00452410842431851\\
45.5	0.00569519120545953\\
46	0.00712545191808037\\
46.5	0.00886187689457257\\
47	0.0109578577195239\\
47.5	0.0134737006063819\\
48	0.0164771232429652\\
48.5	0.0200437310202536\\
49	0.0242574642905188\\
49.5	0.0292110081887073\\
50	0.0350061566037896\\
50.5	0.0417541221091475\\
51	0.0495757840530513\\
51.5	0.0586018675675375\\
52	0.0689730469678089\\
52.5	0.0808399678721332\\
53	0.0943631833574936\\
53.5	0.109713000559963\\
54	0.127069235308519\\
54.5	0.146620873623371\\
55	0.168565640189554\\
55.5	0.193109475208207\\
56	0.220465922305986\\
56.5	0.25085543142332\\
57	0.284504581780979\\
57.5	0.321645231120943\\
58	0.362513598412619\\
58.5	0.407349288092898\\
59	0.45639426465533\\
59.5	0.509891787009454\\
60	0.568085312489308\\
60.5	0.631217380696946\\
61	0.699528487521156\\
61.5	0.773255959676996\\
62	0.852632839972159\\
62.5	0.937886793230243\\
63	1.02923904239796\\
63.5	1.12690334384461\\
64	1.23108501024167\\
64.5	1.34197998870087\\
65	1.45977400106727\\
65.5	1.58464175242322\\
66	1.7167462129764\\
66.5	1.85623797759459\\
67	2.00325470632665\\
67.5	2.15792064832686\\
68	2.32034625069177\\
68.5	2.49062785283604\\
69	2.668847466188\\
69.5	2.85507263818527\\
70	3.0493563988044\\
70.5	3.25173728717177\\
71	3.46223945518204\\
71.5	3.68087284449808\\
72	3.90763343282542\\
72.5	4.14250354494638\\
73	4.38545222366307\\
73.5	4.63643565553483\\
74	4.89539764610137\\
74.5	5.16227013915577\\
75	5.43697377456752\\
75.5	5.71941847915171\\
76	6.00950408513145\\
76.5	6.30712097084087\\
77	6.61215071846256\\
77.5	6.92446678377892\\
78	7.24393517313696\\
78.5	7.57041512307599\\
79	7.90375977834112\\
79.5	8.24381686429867\\
80	8.59042935007648\\
80.5	8.94343609906976\\
81	9.30267250377535\\
81.5	9.66797110224302\\
82	10.0391621737541\\
82.5	10.4160743116574\\
83	10.7985349716013\\
83.5	11.1863709937031\\
84	11.5794090974835\\
84.5	11.9774763486689\\
85	12.3804005972239\\
85.5	12.7880108862167\\
86	13.2001378313502\\
86.5	13.616613971192\\
87	14.0372740883327\\
87.5	14.4619555018702\\
88	14.8904983317697\\
88.5	15.322745735789\\
89	15.7585441197714\\
89.5	16.1977433222165\\
90	16.64019677412\\
90.5	17.0857616351474\\
91	17.534298907261\\
91.5	17.9856735269652\\
92	18.4397544373629\\
92.5	18.8964146412374\\
93	19.3555312363815\\
93.5	19.8169854343963\\
94	20.2806625641717\\
94.5	20.7464520612442\\
95	21.2142474442043\\
95.5	21.6839462792975\\
96	22.1554501343258\\
96.5	22.6286645229214\\
97	23.103498840221\\
97.5	23.5798662909231\\
98	24.0576838106659\\
98.5	24.5368719816149\\
99	25.0173549430976\\
99.5	25.4990602980747\\
100	25.9819190161881\\
100.5	26.465865334072\\
101	26.9508366535693\\
101.5	27.4367734384434\\
102	27.9236191101299\\
102.5	28.4113199430269\\
103	28.8998249597779\\
103.5	29.3890858269596\\
104	29.8790567515443\\
104.5	30.369694378471\\
105	30.8609576896191\\
105.5	31.3528079044457\\
106	31.845208382515\\
106.5	32.3381245281161\\
107	32.831523697138\\
107.5	33.3253751063431\\
108	33.8196497451569\\
108.5	34.3143202900668\\
109	34.809361021704\\
109.5	35.304747744662\\
110	35.8004577100867\\
110.5	36.2964695410608\\
111	36.7927631607856\\
111.5	37.2893197235561\\
112	37.7861215485076\\
112.5	38.283152056108\\
113	38.780395707357\\
113.5	39.2778379456456\\
114	39.775465141225\\
114.5	40.2732645382264\\
115	40.7712242041681\\
115.5	41.2693329818834\\
116	41.7675804437981\\
116.5	42.265956848486\\
117	42.764453099426\\
117.5	43.263060705885\\
118	43.7617717458504\\
118.5	44.2605788309325\\
119	44.7594750731615\\
119.5	45.2584540536004\\
120	45.7575097926983\\
120.5	46.2566367223079\\
121	46.7558296592937\\
121.5	47.2550837806576\\
122	47.7543946001111\\
122.5	48.2537579460249\\
123	48.7531699406881\\
123.5	49.2526269808119\\
124	49.7521257192145\\
124.5	50.2516630476256\\
125	50.7512360805527\\
125.5	51.2508421401503\\
126	51.7504787420398\\
126.5	52.2501435820265\\
127	52.7498345236632\\
127.5	53.2495495866132\\
128	53.7492869357672\\
128.5	54.2490448710688\\
129	54.748821818009\\
129.5	55.2486163187477\\
130	55.7484270238275\\
130.5	56.2482526844406\\
131	56.7480921452178\\
131.5	57.2479443375052\\
132	57.7478082731003\\
132.5	58.2476830384161\\
133	58.7475677890482\\
133.5	59.2474617447184\\
134	59.74736418457\\
134.5	60.2472744427925\\
135	60.7471919045547\\
135.5	61.2471160022241\\
136	61.7470462118557\\
136.5	62.2469820499312\\
137	62.7469230703309\\
137.5	63.246868861525\\
138	63.7468190439664\\
138.5	64.2467732676736\\
139	64.7467312099894\\
139.5	65.2466925735032\\
140	65.7466570841264\\
140.5	66.2466244893094\\
141	66.74659455639\\
141.5	67.2465670710652\\
142	67.7465418359759\\
142.5	68.2465186693983\\
143	68.7464974040324\\
143.5	69.2464778858825\\
144	69.7464599732216\\
144.5	70.2464435356346\\
145	70.7464284531333\\
145.5	71.2464146153402\\
146	71.7464019207329\\
146.5	72.2463902759479\\
147	72.7463795951368\\
147.5	73.2463697993724\\
148	73.7463608161009\\
148.5	74.246352578636\\
149	74.7463450256926\\
149.5	75.2463381009567\\
150	75.7463317526884\\
};
\end{axis}
\end{tikzpicture}%
\caption{Variation du prix du CALL en fonction du sous-jacent $S_0$}
\label{fig:call_s}
\end{figure}

On voit une fonction convexe croissante en fonction de l'evolution de $S_{0}$. On remarque par ailleurs que la courbe tend vers la droite  $(S_{0}-K)_+$.

\vfill

\newpage

\begin{figure}[H]
\centering
% This file was created by matlab2tikz.
%
%The latest updates can be retrieved from
%  http://www.mathworks.com/matlabcentral/fileexchange/22022-matlab2tikz-matlab2tikz
%where you can also make suggestions and rate matlab2tikz.
%
\begin{tikzpicture}

\begin{axis}[%
width=0.6\linewidth,
height=0.6\linewidth,
at={(1.011in,0.642in)},
scale only axis,
xmin=0,
xmax=0.5,
xlabel style={font=\color{white!15!black}},
xlabel={$\sigma$},
ymin=0,
ymax=16,
ylabel style={font=\color{white!15!black}},
ylabel={CALL},
axis background/.style={fill=white}
]
\addplot [color=black, line width=0.8pt, forget plot]
  table[row sep=crcr]{%
0	0.74626246881239\\
0.001	0.74626246881239\\
0.002	0.746262476791642\\
0.003	0.746287560538804\\
0.004	0.746860710267867\\
0.005	0.749430594175749\\
0.006	0.755139888839224\\
0.007	0.764243344272899\\
0.008	0.776463043016619\\
0.009	0.791345899730914\\
0.01	0.808436925198144\\
0.011	0.827339593365011\\
0.012	0.847727179555896\\
0.013	0.869336547244295\\
0.014	0.891957364414637\\
0.015	0.915421497923134\\
0.016	0.939594015053757\\
0.017	0.964365983369404\\
0.018	0.9896488543339\\
0.019	1.01537012668518\\
0.02	1.04147000569575\\
0.021	1.06789882455645\\
0.022	1.09461504534421\\
0.023	1.12158370049652\\
0.024	1.14877516997301\\
0.025	1.17616421538088\\
0.026	1.2037292118989\\
0.027	1.23145153339492\\
0.028	1.25931505695296\\
0.029	1.28730576108205\\
0.03	1.3154113978977\\
0.031	1.34362122408398\\
0.032	1.37192577885259\\
0.033	1.40031669970256\\
0.034	1.42878656875946\\
0.035	1.45732878398898\\
0.036	1.48593745075294\\
0.037	1.51460729008613\\
0.038	1.54333356078458\\
0.039	1.57211199295563\\
0.04	1.60093873112323\\
0.041	1.62981028533267\\
0.042	1.65872348898058\\
0.043	1.68767546232123\\
0.044	1.71666358078188\\
0.045	1.74568544736729\\
0.046	1.77473886855395\\
0.047	1.80382183317165\\
0.048	1.8329324938513\\
0.049	1.86206915068387\\
0.05	1.89123023679011\\
0.051	1.92041430554635\\
0.052	1.9496200192496\\
0.053	1.97884613903684\\
0.054	2.00809151590011\\
0.055	2.03735508266127\\
0.056	2.06663584678936\\
0.057	2.09593288395933\\
0.058	2.12524533226471\\
0.059	2.15457238700821\\
0.06	2.18391329600418\\
0.061	2.21326735533541\\
0.062	2.24263390551378\\
0.063	2.27201232800101\\
0.064	2.30140204205045\\
0.065	2.3308025018362\\
0.066	2.3602131938395\\
0.067	2.38963363446589\\
0.068	2.41906336786987\\
0.069	2.44850196396623\\
0.07	2.47794901660968\\
0.071	2.50740414192642\\
0.072	2.53686697678317\\
0.073	2.56633717738045\\
0.074	2.59581441795878\\
0.075	2.62529838960715\\
0.076	2.65478879916453\\
0.077	2.68428536820625\\
0.078	2.71378783210745\\
0.079	2.74329593917685\\
0.08	2.7728094498552\\
0.081	2.80232813597217\\
0.082	2.83185178005738\\
0.083	2.86138017470066\\
0.084	2.89091312195773\\
0.085	2.92045043279732\\
0.086	2.94999192658669\\
0.087	2.97953743061227\\
0.088	3.00908677963272\\
0.089	3.03863981546198\\
0.09	3.06819638657976\\
0.091	3.09775634776767\\
0.092	3.1273195597688\\
0.093	3.1568858889692\\
0.094	3.18645520709947\\
0.095	3.21602739095514\\
0.096	3.24560232213433\\
0.097	3.27517988679165\\
0.098	3.30475997540695\\
0.099	3.33434248256808\\
0.1	3.36392730676656\\
0.101	3.39351435020537\\
0.102	3.4231035186179\\
0.103	3.45269472109749\\
0.104	3.48228786993665\\
0.105	3.51188288047546\\
0.106	3.54147967095856\\
0.107	3.5710781624\\
0.108	3.60067827845565\\
0.109	3.63027994530259\\
0.11	3.659883091525\\
0.111	3.68948764800625\\
0.112	3.71909354782663\\
0.113	3.74870072616665\\
0.114	3.77830912021534\\
0.115	3.80791866908316\\
0.116	3.83752931371967\\
0.117	3.8671409968352\\
0.118	3.89675366282653\\
0.119	3.92636725770637\\
0.12	3.95598172903625\\
0.121	3.98559702586283\\
0.122	4.01521309865722\\
0.123	4.04482989925737\\
0.124	4.07444738081306\\
0.125	4.10406549773379\\
0.126	4.13368420563881\\
0.127	4.16330346130985\\
0.128	4.19292322264573\\
0.129	4.22254344861933\\
0.13	4.25216409923641\\
0.131	4.28178513549629\\
0.132	4.31140651935444\\
0.133	4.34102821368662\\
0.134	4.37065018225465\\
0.135	4.4002723896738\\
0.136	4.42989480138143\\
0.137	4.45951738360719\\
0.138	4.48914010334436\\
0.139	4.51876292832254\\
0.14	4.54838582698137\\
0.141	4.57800876844555\\
0.142	4.60763172250078\\
0.143	4.63725465957067\\
0.144	4.66687755069476\\
0.145	4.69650036750735\\
0.146	4.72612308221719\\
0.147	4.75574566758799\\
0.148	4.78536809691987\\
0.149	4.81499034403127\\
0.15	4.84461238324184\\
0.151	4.87423418935595\\
0.152	4.9038557376467\\
0.153	4.93347700384078\\
0.154	4.9630979641037\\
0.155	4.99271859502581\\
0.156	5.02233887360863\\
0.157	5.05195877725195\\
0.158	5.08157828374113\\
0.159	5.11119737123516\\
0.16	5.14081601825497\\
0.161	5.17043420367227\\
0.162	5.20005190669877\\
0.163	5.22966910687581\\
0.164	5.25928578406436\\
0.165	5.28890191843536\\
0.166	5.31851749046048\\
0.167	5.34813248090314\\
0.168	5.37774687080986\\
0.169	5.40736064150197\\
0.17	5.43697377456752\\
0.171	5.46658625185357\\
0.172	5.49619805545875\\
0.173	5.52580916772588\\
0.174	5.5554195712352\\
0.175	5.58502924879744\\
0.176	5.61463818344747\\
0.177	5.64424635843792\\
0.178	5.67385375723311\\
0.179	5.70346036350323\\
0.18	5.73306616111863\\
0.181	5.76267113414431\\
0.182	5.79227526683467\\
0.183	5.82187854362832\\
0.184	5.85148094914314\\
0.185	5.88108246817148\\
0.186	5.91068308567552\\
0.187	5.94028278678271\\
0.188	5.96988155678147\\
0.189	5.99947938111699\\
0.19	6.02907624538704\\
0.191	6.05867213533816\\
0.192	6.08826703686167\\
0.193	6.11786093599009\\
0.194	6.14745381889343\\
0.195	6.1770456718758\\
0.196	6.20663648137197\\
0.197	6.23622623394407\\
0.198	6.26581491627847\\
0.199	6.29540251518263\\
0.2	6.3249890175822\\
0.201	6.35457441051802\\
0.202	6.38415868114335\\
0.203	6.41374181672116\\
0.204	6.44332380462141\\
0.205	6.47290463231853\\
0.206	6.50248428738894\\
0.207	6.53206275750853\\
0.208	6.56164003045041\\
0.209	6.59121609408258\\
0.21	6.62079093636563\\
0.211	6.65036454535073\\
0.212	6.6799369091774\\
0.213	6.70950801607153\\
0.214	6.7390778543434\\
0.215	6.76864641238572\\
0.216	6.79821367867178\\
0.217	6.82777964175362\\
0.218	6.85734429026029\\
0.219	6.88690761289604\\
0.22	6.91646959843877\\
0.221	6.94603023573828\\
0.222	6.97558951371477\\
0.223	7.00514742135723\\
0.224	7.034703947722\\
0.225	7.06425908193125\\
0.226	7.09381281317162\\
0.227	7.12336513069277\\
0.228	7.15291602380608\\
0.229	7.1824654818833\\
0.23	7.21201349435531\\
0.231	7.24156005071084\\
0.232	7.27110514049527\\
0.233	7.30064875330945\\
0.234	7.33019087880857\\
0.235	7.35973150670097\\
0.236	7.38927062674713\\
0.237	7.41880822875855\\
0.238	7.44834430259673\\
0.239	7.47787883817216\\
0.24	7.50741182544331\\
0.241	7.5369432544157\\
0.242	7.56647311514095\\
0.243	7.59600139771586\\
0.244	7.62552809228151\\
0.245	7.65505318902241\\
0.246	7.68457667816564\\
0.247	7.71409854998001\\
0.248	7.74361879477528\\
0.249	7.77313740290134\\
0.25	7.80265436474746\\
0.251	7.83216967074154\\
0.252	7.86168331134935\\
0.253	7.89119527707387\\
0.254	7.92070555845451\\
0.255	7.95021414606653\\
0.256	7.97972103052027\\
0.257	8.00922620246056\\
0.258	8.03872965256609\\
0.259	8.06823137154876\\
0.26	8.09773135015308\\
0.261	8.1272295791556\\
0.262	8.1567260493643\\
0.263	8.18622075161805\\
0.264	8.21571367678606\\
0.265	8.2452048157673\\
0.266	8.27469415949005\\
0.267	8.30418169891129\\
0.268	8.33366742501629\\
0.269	8.36315132881801\\
0.27	8.39263340135676\\
0.271	8.42211363369958\\
0.272	8.45159201693991\\
0.273	8.48106854219704\\
0.274	8.51054320061572\\
0.275	8.54001598336573\\
0.276	8.56948688164148\\
0.277	8.5989558866615\\
0.278	8.62842298966819\\
0.279	8.65788818192727\\
0.28	8.68735145472751\\
0.281	8.71681279938033\\
0.282	8.74627220721936\\
0.283	8.77572966960017\\
0.284	8.80518517789989\\
0.285	8.83463872351683\\
0.286	8.86409029787018\\
0.287	8.89353989239966\\
0.288	8.92298749856519\\
0.289	8.95243310784662\\
0.29	8.98187671174338\\
0.291	9.01131830177414\\
0.292	9.04075786947658\\
0.293	9.0701954064071\\
0.294	9.09963090414045\\
0.295	9.12906435426957\\
0.296	9.15849574840519\\
0.297	9.18792507817567\\
0.298	9.21735233522668\\
0.299	9.24677751122096\\
0.3	9.27620059783806\\
0.301	9.30562158677409\\
0.302	9.33504046974151\\
0.303	9.36445723846885\\
0.304	9.39387188470049\\
0.305	9.42328440019648\\
0.306	9.45269477673221\\
0.307	9.48210300609833\\
0.308	9.5115090801004\\
0.309	9.54091299055877\\
0.31	9.57031472930839\\
0.311	9.5997142881985\\
0.312	9.62911165909254\\
0.313	9.6585068338679\\
0.314	9.68789980441574\\
0.315	9.71729056264083\\
0.316	9.74667910046134\\
0.317	9.77606540980865\\
0.318	9.80544948262723\\
0.319	9.83483131087437\\
0.32	9.86421088652015\\
0.321	9.89358820154713\\
0.322	9.92296324795027\\
0.323	9.95233601773678\\
0.324	9.9817065029259\\
0.325	10.0110746955488\\
0.326	10.0404405876484\\
0.327	10.0698041712793\\
0.328	10.0991654385076\\
0.329	10.1285243814104\\
0.33	10.1578809920765\\
0.331	10.1872352626054\\
0.332	10.2165871851075\\
0.333	10.2459367517043\\
0.334	10.2752839545277\\
0.335	10.3046287857201\\
0.336	10.3339712374346\\
0.337	10.3633113018342\\
0.338	10.3926489710923\\
0.339	10.4219842373923\\
0.34	10.4513170929276\\
0.341	10.4806475299013\\
0.342	10.5099755405263\\
0.343	10.539301117025\\
0.344	10.5686242516294\\
0.345	10.5979449365807\\
0.346	10.6272631641297\\
0.347	10.6565789265361\\
0.348	10.6858922160689\\
0.349	10.7152030250057\\
0.35	10.7445113456335\\
0.351	10.7738171702478\\
0.352	10.8031204911527\\
0.353	10.8324213006612\\
0.354	10.8617195910945\\
0.355	10.8910153547826\\
0.356	10.9203085840635\\
0.357	10.9495992712838\\
0.358	10.9788874087978\\
0.359	11.0081729889684\\
0.36	11.0374560041662\\
0.361	11.0667364467698\\
0.362	11.0960143091658\\
0.363	11.1252895837484\\
0.364	11.1545622629195\\
0.365	11.1838323390889\\
0.366	11.2130998046737\\
0.367	11.2423646520985\\
0.368	11.2716268737955\\
0.369	11.300886462204\\
0.37	11.3301434097709\\
0.371	11.35939770895\\
0.372	11.3886493522025\\
0.373	11.4178983319966\\
0.374	11.4471446408074\\
0.375	11.4763882711172\\
0.376	11.505629215415\\
0.377	11.5348674661969\\
0.378	11.5641030159655\\
0.379	11.5933358572303\\
0.38	11.6225659825074\\
0.381	11.6517933843195\\
0.382	11.6810180551961\\
0.383	11.7102399876728\\
0.384	11.739459174292\\
0.385	11.7686756076024\\
0.386	11.797889280159\\
0.387	11.8271001845232\\
0.388	11.8563083132624\\
0.389	11.8855136589507\\
0.39	11.9147162141678\\
0.391	11.9439159714999\\
0.392	11.9731129235391\\
0.393	12.0023070628836\\
0.394	12.0314983821374\\
0.395	12.0606868739108\\
0.396	12.0898725308195\\
0.397	12.1190553454856\\
0.398	12.1482353105364\\
0.399	12.1774124186054\\
0.4	12.2065866623317\\
0.401	12.2357580343601\\
0.402	12.264926527341\\
0.403	12.2940921339304\\
0.404	12.32325484679\\
0.405	12.3524146585868\\
0.406	12.3815715619936\\
0.407	12.4107255496883\\
0.408	12.4398766143545\\
0.409	12.4690247486812\\
0.41	12.4981699453625\\
0.411	12.5273121970982\\
0.412	12.556451496593\\
0.413	12.5855878365571\\
0.414	12.6147212097059\\
0.415	12.6438516087599\\
0.416	12.6729790264449\\
0.417	12.7021034554917\\
0.418	12.7312248886363\\
0.419	12.7603433186198\\
0.42	12.7894587381882\\
0.421	12.8185711400927\\
0.422	12.8476805170893\\
0.423	12.8767868619392\\
0.424	12.9058901674082\\
0.425	12.9349904262675\\
0.426	12.9640876312927\\
0.427	12.9931817752645\\
0.428	13.0222728509684\\
0.429	13.0513608511947\\
0.43	13.0804457687385\\
0.431	13.1095275963997\\
0.432	13.1386063269828\\
0.433	13.1676819532971\\
0.434	13.1967544681567\\
0.435	13.2258238643802\\
0.436	13.2548901347909\\
0.437	13.2839532722168\\
0.438	13.3130132694904\\
0.439	13.3420701194488\\
0.44	13.3711238149337\\
0.441	13.4001743487914\\
0.442	13.4292217138726\\
0.443	13.4582659030327\\
0.444	13.4873069091312\\
0.445	13.5163447250325\\
0.446	13.5453793436052\\
0.447	13.5744107577225\\
0.448	13.6034389602618\\
0.449	13.6324639441049\\
0.45	13.6614857021383\\
0.451	13.6905042272525\\
0.452	13.7195195123424\\
0.453	13.7485315503075\\
0.454	13.7775403340513\\
0.455	13.8065458564818\\
0.456	13.835548110511\\
0.457	13.8645470890555\\
0.458	13.8935427850359\\
0.459	13.9225351913773\\
0.46	13.9515243010087\\
0.461	13.9805101068635\\
0.462	14.0094926018792\\
0.463	14.0384717789976\\
0.464	14.0674476311645\\
0.465	14.09642015133\\
0.466	14.1253893324482\\
0.467	14.1543551674775\\
0.468	14.1833176493802\\
0.469	14.2122767711228\\
0.47	14.241232525676\\
0.471	14.2701849060144\\
0.472	14.2991339051167\\
0.473	14.3280795159657\\
0.474	14.3570217315481\\
0.475	14.3859605448549\\
0.476	14.414895948881\\
0.477	14.443827936625\\
0.478	14.47275650109\\
0.479	14.5016816352828\\
0.48	14.5306033322141\\
0.481	14.5595215848988\\
0.482	14.5884363863556\\
0.483	14.6173477296072\\
0.484	14.6462556076802\\
0.485	14.6751600136051\\
0.486	14.7040609404165\\
0.487	14.7329583811526\\
0.488	14.7618523288558\\
0.489	14.7907427765722\\
0.49	14.8196297173518\\
0.491	14.8485131442486\\
0.492	14.8773930503203\\
0.493	14.9062694286286\\
0.494	14.935142272239\\
0.495	14.9640115742208\\
0.496	14.9928773276471\\
0.497	15.021739525595\\
0.498	15.0505981611453\\
0.499	15.0794532273825\\
0.5	15.1083047173951\\
};
\end{axis}
\end{tikzpicture}%
\caption{Variation du prix du CALL en fonction de la volatilité $\sigma$}
\label{fig:call_sigma}
\end{figure}

En faisant varier la volatilité $\sigma$, on obtient un graphe qui semble linéaire à partir d'une certaine valeur. La fonction est croissante ce qui est lié au fait que plus la volitilité est élévée plus les gains potentiels sont importants (il ne faut pas oublier que les investisseurs sont {\slshape risk-averse}). Néanmoins, si la volatilité du prix du sous jacente est nulle, le détenteur de l'option n'a aucune perspective de gains et revend son option à un prix "nul" (ce qui concorde avec le graphique).  

\begin{figure}[H]
\centering
% This file was created by matlab2tikz.
%
%The latest updates can be retrieved from
%  http://www.mathworks.com/matlabcentral/fileexchange/22022-matlab2tikz-matlab2tikz
%where you can also make suggestions and rate matlab2tikz.
%
\begin{tikzpicture}

\begin{axis}[%
width=0.6\linewidth,
height=0.6\linewidth,
at={(1.011in,0.642in)},
scale only axis,
unbounded coords=jump,
xmin=0,
xmax=2,
xlabel style={font=\color{white!15!black}},
xlabel={T},
ymin=0,
ymax=8,
ylabel style={font=\color{white!15!black}},
ylabel={CALL},
axis background/.style={fill=white}
]
\addplot [color=black, line width=0.8pt, forget plot]
  table[row sep=crcr]{%
0	nan\\
0.001	0.161223976754847\\
0.002	0.228223781842644\\
0.003	0.279721093548858\\
0.004	0.323193609431804\\
0.005	0.361537766839504\\
0.006	0.396239042128904\\
0.007	0.428179880450259\\
0.008	0.45793524625676\\
0.009	0.485904513032644\\
0.01	0.512378463495303\\
0.011	0.537576616098569\\
0.012	0.561669515613048\\
0.013	0.584792787952189\\
0.014	0.607056398579559\\
0.015	0.628550975211276\\
0.016	0.649352258222251\\
0.017	0.669524314855103\\
0.018	0.689121912685707\\
0.019	0.708192306459615\\
0.02	0.726776606342085\\
0.021	0.744910841535621\\
0.022	0.762626798275733\\
0.023	0.779952688081764\\
0.024	0.796913686486484\\
0.025	0.81353237166649\\
0.026	0.829829084808409\\
0.027	0.845822228630887\\
0.028	0.861528516559765\\
0.029	0.87696318217484\\
0.03	0.892140156406107\\
0.031	0.907072218349086\\
0.032	0.921771124346463\\
0.033	0.93624771904571\\
0.034	0.950512031416423\\
0.035	0.964573358144548\\
0.036	0.978440336374334\\
0.037	0.992121007415236\\
0.038	1.00562287274892\\
0.039	1.01895294344378\\
0.04	1.03211778390165\\
0.041	1.04512355071141\\
0.042	1.05797602726219\\
0.043	1.07068065466866\\
0.044	1.08324255947734\\
0.045	1.09566657855439\\
0.046	1.10795728149752\\
0.047	1.12011899086655\\
0.048	1.13215580048672\\
0.049	1.14407159204444\\
0.05	1.15587005016623\\
0.051	1.16755467614693\\
0.052	1.1791288004723\\
0.053	1.19059559426272\\
0.054	1.20195807974974\\
0.055	1.21321913988323\\
0.056	1.22438152715607\\
0.057	1.23544787172263\\
0.058	1.24642068887895\\
0.059	1.257302385965\\
0.06	1.26809526874266\\
0.061	1.27880154729715\\
0.062	1.28942334150491\\
0.063	1.29996268610607\\
0.064	1.31042153541606\\
0.065	1.3208017677071\\
0.066	1.33110518928751\\
0.067	1.34133353830389\\
0.068	1.35148848828891\\
0.069	1.36157165147497\\
0.07	1.37158458189278\\
0.071	1.38152877827113\\
0.072	1.39140568675371\\
0.073	1.40121670344666\\
0.074	1.41096317680951\\
0.075	1.42064640990135\\
0.076	1.43026766249246\\
0.077	1.43982815305151\\
0.078	1.44932906061686\\
0.079	1.45877152656033\\
0.08	1.4681566562507\\
0.081	1.47748552062401\\
0.082	1.48675915766692\\
0.083	1.4959785738187\\
0.084	1.50514474529775\\
0.085	1.51425861935692\\
0.086	1.52332111547289\\
0.087	1.53233312647319\\
0.088	1.54129551960536\\
0.089	1.55020913755135\\
0.09	1.55907479939098\\
0.091	1.56789330151719\\
0.092	1.57666541850627\\
0.093	1.58539190394554\\
0.094	1.5940734912212\\
0.095	1.60271089426846\\
0.096	1.61130480828628\\
0.097	1.61985591041865\\
0.098	1.62836486040445\\
0.099	1.63683230119742\\
0.1	1.64525885955804\\
0.101	1.65364514661898\\
0.102	1.66199175842515\\
0.103	1.67029927645028\\
0.104	1.67856826809079\\
0.105	1.68679928713843\\
0.106	1.69499287423262\\
0.107	1.70314955729371\\
0.108	1.71126985193807\\
0.109	1.71935426187592\\
0.11	1.72740327929269\\
0.111	1.73541738521486\\
0.112	1.74339704986112\\
0.113	1.75134273297922\\
0.114	1.75925488416966\\
0.115	1.7671339431965\\
0.116	1.77498034028603\\
0.117	1.78279449641396\\
0.118	1.79057682358159\\
0.119	1.79832772508129\\
0.12	1.80604759575221\\
0.121	1.8137368222263\\
0.122	1.82139578316513\\
0.123	1.82902484948814\\
0.124	1.83662438459248\\
0.125	1.84419474456485\\
0.126	1.8517362783858\\
0.127	1.85924932812674\\
0.128	1.86673422913994\\
0.129	1.87419131024188\\
0.13	1.88162089389024\\
0.131	1.88902329635474\\
0.132	1.8963988278822\\
0.133	1.90374779285582\\
0.134	1.91107048994929\\
0.135	1.91836721227558\\
0.136	1.92563824753083\\
0.137	1.93288387813349\\
0.138	1.94010438135891\\
0.139	1.94730002946955\\
0.14	1.95447108984089\\
0.141	1.9616178250835\\
0.142	1.96874049316099\\
0.143	1.9758393475044\\
0.144	1.98291463712286\\
0.145	1.98996660671103\\
0.146	1.99699549675287\\
0.147	2.00400154362249\\
0.148	2.01098497968185\\
0.149	2.01794603337542\\
0.15	2.02488492932208\\
0.151	2.03180188840421\\
0.152	2.03869712785413\\
0.153	2.04557086133804\\
0.154	2.05242329903746\\
0.155	2.0592546477282\\
0.156	2.06606511085725\\
0.157	2.07285488861731\\
0.158	2.07962417801917\\
0.159	2.08637317296207\\
0.16	2.09310206430217\\
0.161	2.09981103991893\\
0.162	2.1065002847797\\
0.163	2.11316998100261\\
0.164	2.11982030791758\\
0.165	2.12645144212577\\
0.166	2.13306355755736\\
0.167	2.13965682552773\\
0.168	2.14623141479223\\
0.169	2.15278749159945\\
0.17	2.15932521974295\\
0.171	2.16584476061183\\
0.172	2.17234627323979\\
0.173	2.17882991435307\\
0.174	2.18529583841698\\
0.175	2.19174419768132\\
0.176	2.19817514222468\\
0.177	2.20458881999749\\
0.178	2.21098537686405\\
0.179	2.21736495664355\\
0.18	2.22372770114985\\
0.181	2.23007375023062\\
0.182	2.23640324180509\\
0.183	2.24271631190123\\
0.184	2.24901309469185\\
0.185	2.25529372252974\\
0.186	2.26155832598216\\
0.187	2.26780703386434\\
0.188	2.27403997327223\\
0.189	2.28025726961447\\
0.19	2.28645904664355\\
0.191	2.29264542648637\\
0.192	2.29881652967384\\
0.193	2.30497247517011\\
0.194	2.31111338040078\\
0.195	2.3172393612807\\
0.196	2.32335053224106\\
0.197	2.3294470062558\\
0.198	2.33552889486746\\
0.199	2.34159630821252\\
0.2	2.34764935504602\\
0.201	2.35368814276573\\
0.202	2.35971277743576\\
0.203	2.36572336380959\\
0.204	2.37172000535273\\
0.205	2.3777028042647\\
0.206	2.38367186150062\\
0.207	2.3896272767924\\
0.208	2.39556914866929\\
0.209	2.40149757447819\\
0.21	2.40741265040335\\
0.211	2.41331447148575\\
0.212	2.41920313164203\\
0.213	2.42507872368308\\
0.214	2.43094133933207\\
0.215	2.43679106924233\\
0.216	2.44262800301472\\
0.217	2.44845222921459\\
0.218	2.45426383538854\\
0.219	2.4600629080807\\
0.22	2.46584953284883\\
0.221	2.47162379427982\\
0.222	2.47738577600524\\
0.223	2.48313556071619\\
0.224	2.48887323017821\\
0.225	2.49459886524565\\
0.226	2.50031254587578\\
0.227	2.50601435114276\\
0.228	2.51170435925116\\
0.229	2.51738264754935\\
0.23	2.52304929254255\\
0.231	2.5287043699056\\
0.232	2.5343479544956\\
0.233	2.5399801203642\\
0.234	2.54560094076962\\
0.235	2.55121048818865\\
0.236	2.55680883432813\\
0.237	2.56239605013643\\
0.238	2.56797220581462\\
0.239	2.57353737082745\\
0.24	2.57909161391407\\
0.241	2.58463500309871\\
0.242	2.59016760570096\\
0.243	2.59568948834597\\
0.244	2.60120071697442\\
0.245	2.60670135685238\\
0.246	2.6121914725809\\
0.247	2.61767112810546\\
0.248	2.62314038672528\\
0.249	2.62859931110234\\
0.25	2.63404796327045\\
0.251	2.63948640464397\\
0.252	2.64491469602642\\
0.253	2.65033289761897\\
0.254	2.65574106902879\\
0.255	2.66113926927721\\
0.256	2.66652755680773\\
0.257	2.67190598949391\\
0.258	2.6772746246472\\
0.259	2.68263351902443\\
0.26	2.6879827288354\\
0.261	2.6933223097502\\
0.262	2.6986523169064\\
0.263	2.70397280491625\\
0.264	2.70928382787356\\
0.265	2.71458543936065\\
0.266	2.71987769245501\\
0.267	2.72516063973602\\
0.268	2.73043433329144\\
0.269	2.7356988247238\\
0.27	2.74095416515674\\
0.271	2.7462004052412\\
0.272	2.75143759516149\\
0.273	2.75666578464133\\
0.274	2.76188502294975\\
0.275	2.76709535890684\\
0.276	2.7722968408895\\
0.277	2.77748951683707\\
0.278	2.7826734342568\\
0.279	2.78784864022929\\
0.28	2.79301518141391\\
0.281	2.79817310405396\\
0.282	2.80332245398194\\
0.283	2.80846327662456\\
0.284	2.8135956170078\\
0.285	2.81871951976183\\
0.286	2.82383502912587\\
0.287	2.82894218895296\\
0.288	2.83404104271471\\
0.289	2.83913163350584\\
0.29	2.84421400404883\\
0.291	2.84928819669834\\
0.292	2.85435425344568\\
0.293	2.85941221592315\\
0.294	2.86446212540827\\
0.295	2.86950402282807\\
0.296	2.87453794876323\\
0.297	2.87956394345216\\
0.298	2.88458204679497\\
0.299	2.88959229835756\\
0.3	2.89459473737542\\
0.301	2.89958940275757\\
0.302	2.90457633309025\\
0.303	2.90955556664077\\
0.304	2.91452714136109\\
0.305	2.91949109489151\\
0.306	2.92444746456422\\
0.307	2.92939628740683\\
0.308	2.93433760014582\\
0.309	2.93927143920997\\
0.31	2.94419784073377\\
0.311	2.94911684056062\\
0.312	2.95402847424626\\
0.313	2.95893277706186\\
0.314	2.96382978399729\\
0.315	2.96871952976419\\
0.316	2.97360204879909\\
0.317	2.97847737526646\\
0.318	2.9833455430617\\
0.319	2.98820658581407\\
0.32	2.99306053688965\\
0.321	2.99790742939421\\
0.322	3.00274729617605\\
0.323	3.00758016982873\\
0.324	3.01240608269391\\
0.325	3.01722506686404\\
0.326	3.02203715418503\\
0.327	3.02684237625891\\
0.328	3.03164076444641\\
0.329	3.03643234986958\\
0.33	3.04121716341427\\
0.331	3.04599523573269\\
0.332	3.05076659724583\\
0.333	3.05553127814592\\
0.334	3.06028930839884\\
0.335	3.06504071774643\\
0.336	3.06978553570895\\
0.337	3.07452379158724\\
0.338	3.0792555144651\\
0.339	3.0839807332115\\
0.34	3.08869947648279\\
0.341	3.09341177272491\\
0.342	3.09811765017552\\
0.343	3.10281713686614\\
0.344	3.10751026062424\\
0.345	3.11219704907536\\
0.346	3.11687752964511\\
0.347	3.12155172956118\\
0.348	3.1262196758554\\
0.349	3.13088139536565\\
0.35	3.13553691473779\\
0.351	3.14018626042763\\
0.352	3.14482945870284\\
0.353	3.14946653564469\\
0.354	3.15409751715008\\
0.355	3.15872242893321\\
0.356	3.16334129652746\\
0.357	3.16795414528712\\
0.358	3.17256100038919\\
0.359	3.17716188683508\\
0.36	3.18175682945233\\
0.361	3.18634585289631\\
0.362	3.19092898165183\\
0.363	3.1955062400349\\
0.364	3.20007765219422\\
0.365	3.20464324211292\\
0.366	3.20920303361002\\
0.367	3.21375705034211\\
0.368	3.21830531580483\\
0.369	3.22284785333439\\
0.37	3.22738468610916\\
0.371	3.23191583715108\\
0.372	3.23644132932714\\
0.373	3.2409611853509\\
0.374	3.24547542778387\\
0.375	3.24998407903696\\
0.376	3.25448716137183\\
0.377	3.25898469690241\\
0.378	3.26347670759611\\
0.379	3.26796321527526\\
0.38	3.27244424161844\\
0.381	3.27691980816181\\
0.382	3.28138993630041\\
0.383	3.28585464728943\\
0.384	3.2903139622455\\
0.385	3.29476790214797\\
0.386	3.29921648784016\\
0.387	3.30365974003053\\
0.388	3.30809767929399\\
0.389	3.31253032607302\\
0.39	3.31695770067895\\
0.391	3.321379823293\\
0.392	3.32579671396763\\
0.393	3.33020839262748\\
0.394	3.3346148790707\\
0.395	3.33901619296994\\
0.396	3.34341235387352\\
0.397	3.34780338120651\\
0.398	3.35218929427182\\
0.399	3.3565701122513\\
0.4	3.36094585420673\\
0.401	3.36531653908094\\
0.402	3.36968218569881\\
0.403	3.37404281276832\\
0.404	3.37839843888156\\
0.405	3.38274908251569\\
0.406	3.38709476203401\\
0.407	3.39143549568685\\
0.408	3.3957713016126\\
0.409	3.40010219783873\\
0.41	3.40442820228257\\
0.411	3.40874933275238\\
0.412	3.41306560694825\\
0.413	3.41737704246301\\
0.414	3.42168365678313\\
0.415	3.42598546728964\\
0.416	3.43028249125901\\
0.417	3.434574745864\\
0.418	3.43886224817459\\
0.419	3.44314501515876\\
0.42	3.44742306368343\\
0.421	3.45169641051525\\
0.422	3.4559650723214\\
0.423	3.4602290656705\\
0.424	3.46448840703336\\
0.425	3.46874311278382\\
0.426	3.47299319919951\\
0.427	3.4772386824627\\
0.428	3.48147957866097\\
0.429	3.48571590378815\\
0.43	3.48994767374494\\
0.431	3.49417490433973\\
0.432	3.49839761128934\\
0.433	3.50261581021975\\
0.434	3.50682951666688\\
0.435	3.51103874607722\\
0.436	3.51524351380863\\
0.437	3.51944383513107\\
0.438	3.52363972522721\\
0.439	3.5278311991932\\
0.44	3.53201827203934\\
0.441	3.53620095869077\\
0.442	3.54037927398808\\
0.443	3.54455323268812\\
0.444	3.54872284946452\\
0.445	3.55288813890839\\
0.446	3.55704911552904\\
0.447	3.5612057937545\\
0.448	3.56535818793225\\
0.449	3.5695063123298\\
0.45	3.57365018113533\\
0.451	3.57778980845832\\
0.452	3.58192520833013\\
0.453	3.58605639470459\\
0.454	3.59018338145873\\
0.455	3.59430618239313\\
0.456	3.59842481123276\\
0.457	3.60253928162739\\
0.458	3.60664960715221\\
0.459	3.61075580130846\\
0.46	3.61485787752394\\
0.461	3.61895584915352\\
0.462	3.6230497294798\\
0.463	3.62713953171359\\
0.464	3.63122526899443\\
0.465	3.63530695439122\\
0.466	3.63938460090264\\
0.467	3.64345822145778\\
0.468	3.64752782891659\\
0.469	3.65159343607041\\
0.47	3.65565505564248\\
0.471	3.6597127002885\\
0.472	3.66376638259705\\
0.473	3.66781611509014\\
0.474	3.67186191022369\\
0.475	3.675903780388\\
0.476	3.67994173790827\\
0.477	3.68397579504504\\
0.478	3.68800596399468\\
0.479	3.69203225688989\\
0.48	3.69605468580008\\
0.481	3.70007326273193\\
0.482	3.70408799962978\\
0.483	3.70809890837614\\
0.484	3.71210600079203\\
0.485	3.7161092886376\\
0.486	3.72010878361236\\
0.487	3.72410449735578\\
0.488	3.72809644144765\\
0.489	3.73208462740853\\
0.49	3.73606906670013\\
0.491	3.74004977072582\\
0.492	3.74402675083091\\
0.493	3.74800001830324\\
0.494	3.75196958437341\\
0.495	3.7559354602153\\
0.496	3.75989765694641\\
0.497	3.76385618562832\\
0.498	3.76781105726705\\
0.499	3.77176228281336\\
0.5	3.77570987316335\\
0.501	3.77965383915862\\
0.502	3.78359419158679\\
0.503	3.7875309411818\\
0.504	3.79146409862437\\
0.505	3.79539367454226\\
0.506	3.79931967951072\\
0.507	3.80324212405284\\
0.508	3.80716101863983\\
0.509	3.81107637369153\\
0.51	3.8149881995766\\
0.511	3.81889650661297\\
0.512	3.82280130506818\\
0.513	3.8267026051597\\
0.514	3.83060041705524\\
0.515	3.8344947508732\\
0.516	3.83838561668284\\
0.517	3.84227302450479\\
0.518	3.84615698431124\\
0.519	3.85003750602633\\
0.52	3.85391459952648\\
0.521	3.8577882746407\\
0.522	3.86165854115085\\
0.523	3.86552540879207\\
0.524	3.86938888725299\\
0.525	3.87324898617612\\
0.526	3.87710571515809\\
0.527	3.88095908374999\\
0.528	3.88480910145768\\
0.529	3.88865577774206\\
0.53	3.89249912201936\\
0.531	3.89633914366151\\
0.532	3.90017585199635\\
0.533	3.90400925630795\\
0.534	3.9078393658369\\
0.535	3.9116661897806\\
0.536	3.91548973729349\\
0.537	3.91931001748745\\
0.538	3.92312703943193\\
0.539	3.92694081215433\\
0.54	3.93075134464023\\
0.541	3.93455864583368\\
0.542	3.93836272463746\\
0.543	3.94216358991333\\
0.544	3.94596125048231\\
0.545	3.94975571512494\\
0.546	3.95354699258158\\
0.547	3.95733509155257\\
0.548	3.96112002069857\\
0.549	3.96490178864079\\
0.55	3.96868040396123\\
0.551	3.97245587520297\\
0.552	3.97622821087032\\
0.553	3.9799974194292\\
0.554	3.98376350930726\\
0.555	3.9875264888942\\
0.556	3.99128636654198\\
0.557	3.99504315056507\\
0.558	3.99879684924065\\
0.559	4.0025474708089\\
0.56	4.00629502347319\\
0.561	4.01003951540031\\
0.562	4.01378095472072\\
0.563	4.01751934952876\\
0.564	4.02125470788291\\
0.565	4.02498703780592\\
0.566	4.02871634728517\\
0.567	4.03244264427275\\
0.568	4.03616593668576\\
0.569	4.03988623240651\\
0.57	4.04360353928273\\
0.571	4.04731786512777\\
0.572	4.05102921772085\\
0.573	4.05473760480719\\
0.574	4.0584430340983\\
0.575	4.06214551327218\\
0.576	4.06584504997341\\
0.577	4.06954165181355\\
0.578	4.07323532637115\\
0.579	4.07692608119206\\
0.58	4.08061392378958\\
0.581	4.0842988616447\\
0.582	4.08798090220624\\
0.583	4.0916600528911\\
0.584	4.09533632108439\\
0.585	4.09900971413969\\
0.586	4.10268023937916\\
0.587	4.10634790409379\\
0.588	4.11001271554358\\
0.589	4.11367468095768\\
0.59	4.11733380753463\\
0.591	4.12099010244248\\
0.592	4.12464357281903\\
0.593	4.12829422577197\\
0.594	4.13194206837908\\
0.595	4.13558710768838\\
0.596	4.13922935071836\\
0.597	4.14286880445807\\
0.598	4.14650547586732\\
0.599	4.15013937187695\\
0.6	4.15377049938884\\
0.601	4.1573988652762\\
0.602	4.16102447638366\\
0.603	4.16464733952749\\
0.604	4.16826746149574\\
0.605	4.17188484904841\\
0.606	4.17549950891758\\
0.607	4.17911144780764\\
0.608	4.18272067239539\\
0.609	4.1863271893302\\
0.61	4.18993100523422\\
0.611	4.1935321267025\\
0.612	4.1971305603031\\
0.613	4.20072631257734\\
0.614	4.20431939003988\\
0.615	4.20790979917889\\
0.616	4.21149754645624\\
0.617	4.21508263830759\\
0.618	4.21866508114254\\
0.619	4.22224488134484\\
0.62	4.22582204527248\\
0.621	4.22939657925785\\
0.622	4.23296848960791\\
0.623	4.23653778260428\\
0.624	4.24010446450341\\
0.625	4.24366854153676\\
0.626	4.24723001991084\\
0.627	4.25078890580749\\
0.628	4.25434520538388\\
0.629	4.25789892477271\\
0.63	4.2614500700824\\
0.631	4.2649986473971\\
0.632	4.26854466277691\\
0.633	4.27208812225804\\
0.634	4.27562903185282\\
0.635	4.27916739754996\\
0.636	4.28270322531463\\
0.637	4.28623652108859\\
0.638	4.28976729079028\\
0.639	4.29329554031503\\
0.64	4.29682127553509\\
0.641	4.30034450229986\\
0.642	4.30386522643594\\
0.643	4.30738345374728\\
0.644	4.31089919001529\\
0.645	4.31441244099896\\
0.646	4.31792321243505\\
0.647	4.32143151003807\\
0.648	4.32493733950055\\
0.649	4.32844070649308\\
0.65	4.33194161666444\\
0.651	4.33544007564167\\
0.652	4.33893608903029\\
0.653	4.34242966241436\\
0.654	4.34592080135658\\
0.655	4.3494095113984\\
0.656	4.3528957980602\\
0.657	4.35637966684129\\
0.658	4.35986112322016\\
0.659	4.36334017265446\\
0.66	4.36681682058121\\
0.661	4.37029107241683\\
0.662	4.37376293355733\\
0.663	4.37723240937832\\
0.664	4.38069950523523\\
0.665	4.38416422646332\\
0.666	4.38762657837784\\
0.667	4.3910865662741\\
0.668	4.39454419542764\\
0.669	4.39799947109424\\
0.67	4.40145239851007\\
0.671	4.40490298289185\\
0.672	4.40835122943684\\
0.673	4.41179714332301\\
0.674	4.41524072970913\\
0.675	4.41868199373486\\
0.676	4.42212094052086\\
0.677	4.4255575751689\\
0.678	4.4289919027619\\
0.679	4.43242392836412\\
0.68	4.43585365702113\\
0.681	4.43928109376007\\
0.682	4.44270624358959\\
0.683	4.44612911150003\\
0.684	4.44954970246351\\
0.685	4.45296802143397\\
0.686	4.4563840733473\\
0.687	4.4597978631215\\
0.688	4.46320939565661\\
0.689	4.46661867583494\\
0.69	4.47002570852114\\
0.691	4.4734304985622\\
0.692	4.47683305078766\\
0.693	4.48023337000959\\
0.694	4.48363146102277\\
0.695	4.48702732860472\\
0.696	4.4904209775158\\
0.697	4.4938124124993\\
0.698	4.49720163828153\\
0.699	4.50058865957188\\
0.7	4.50397348106296\\
0.701	4.50735610743062\\
0.702	4.51073654333409\\
0.703	4.514114793416\\
0.704	4.51749086230249\\
0.705	4.52086475460336\\
0.706	4.52423647491202\\
0.707	4.5276060278057\\
0.708	4.53097341784541\\
0.709	4.53433864957614\\
0.71	4.53770172752686\\
0.711	4.54106265621058\\
0.712	4.54442144012455\\
0.713	4.54777808375017\\
0.714	4.55113259155321\\
0.715	4.55448496798381\\
0.716	4.55783521747656\\
0.717	4.56118334445063\\
0.718	4.56452935330977\\
0.719	4.56787324844243\\
0.72	4.57121503422185\\
0.721	4.57455471500611\\
0.722	4.57789229513814\\
0.723	4.58122777894594\\
0.724	4.58456117074252\\
0.725	4.58789247482606\\
0.726	4.59122169547991\\
0.727	4.59454883697268\\
0.728	4.59787390355839\\
0.729	4.60119689947641\\
0.73	4.60451782895164\\
0.731	4.60783669619452\\
0.732	4.61115350540108\\
0.733	4.61446826075312\\
0.734	4.61778096641813\\
0.735	4.62109162654949\\
0.736	4.62440024528641\\
0.737	4.62770682675411\\
0.738	4.63101137506383\\
0.739	4.6343138943129\\
0.74	4.63761438858484\\
0.741	4.64091286194936\\
0.742	4.64420931846247\\
0.743	4.64750376216657\\
0.744	4.65079619709044\\
0.745	4.65408662724936\\
0.746	4.65737505664518\\
0.747	4.66066148926634\\
0.748	4.66394592908793\\
0.749	4.66722838007183\\
0.75	4.6705088461667\\
0.751	4.67378733130806\\
0.752	4.67706383941832\\
0.753	4.68033837440689\\
0.754	4.68361094017029\\
0.755	4.68688154059202\\
0.756	4.69015017954285\\
0.757	4.6934168608807\\
0.758	4.69668158845082\\
0.759	4.69994436608578\\
0.76	4.70320519760553\\
0.761	4.7064640868175\\
0.762	4.70972103751663\\
0.763	4.71297605348543\\
0.764	4.71622913849403\\
0.765	4.71948029630028\\
0.766	4.72272953064969\\
0.767	4.72597684527565\\
0.768	4.72922224389937\\
0.769	4.73246573022998\\
0.77	4.73570730796451\\
0.771	4.73894698078813\\
0.772	4.74218475237395\\
0.773	4.7454206263833\\
0.774	4.74865460646564\\
0.775	4.7518866962587\\
0.776	4.75511689938845\\
0.777	4.75834521946923\\
0.778	4.76157166010378\\
0.779	4.76479622488325\\
0.78	4.7680189173873\\
0.781	4.77123974118415\\
0.782	4.77445869983062\\
0.783	4.77767579687213\\
0.784	4.78089103584288\\
0.785	4.78410442026576\\
0.786	4.78731595365247\\
0.787	4.79052563950356\\
0.788	4.79373348130853\\
0.789	4.79693948254575\\
0.79	4.80014364668262\\
0.791	4.80334597717561\\
0.792	4.80654647747026\\
0.793	4.80974515100127\\
0.794	4.8129420011925\\
0.795	4.81613703145708\\
0.796	4.81933024519742\\
0.797	4.82252164580526\\
0.798	4.8257112366617\\
0.799	4.82889902113729\\
0.8	4.83208500259207\\
0.801	4.83526918437555\\
0.802	4.83845156982684\\
0.803	4.84163216227465\\
0.804	4.84481096503735\\
0.805	4.847987981423\\
0.806	4.85116321472942\\
0.807	4.85433666824419\\
0.808	4.85750834524477\\
0.809	4.86067824899845\\
0.81	4.86384638276247\\
0.811	4.86701274978403\\
0.812	4.87017735330035\\
0.813	4.87334019653864\\
0.814	4.87650128271631\\
0.815	4.87966061504082\\
0.816	4.88281819670984\\
0.817	4.88597403091124\\
0.818	4.8891281208232\\
0.819	4.89228046961416\\
0.82	4.89543108044293\\
0.821	4.89857995645868\\
0.822	4.90172710080104\\
0.823	4.90487251660009\\
0.824	4.90801620697642\\
0.825	4.91115817504119\\
0.826	4.91429842389611\\
0.827	4.91743695663356\\
0.828	4.92057377633656\\
0.829	4.92370888607886\\
0.83	4.92684228892493\\
0.831	4.92997398793005\\
0.832	4.93310398614033\\
0.833	4.93623228659273\\
0.834	4.9393588923151\\
0.835	4.94248380632626\\
0.836	4.94560703163599\\
0.837	4.94872857124507\\
0.838	4.95184842814539\\
0.839	4.95496660531987\\
0.84	4.95808310574258\\
0.841	4.96119793237877\\
0.842	4.96431108818486\\
0.843	4.96742257610855\\
0.844	4.97053239908878\\
0.845	4.97364056005581\\
0.846	4.97674706193124\\
0.847	4.97985190762809\\
0.848	4.98295510005075\\
0.849	4.98605664209507\\
0.85	4.98915653664844\\
0.851	4.99225478658969\\
0.852	4.99535139478927\\
0.853	4.99844636410923\\
0.854	5.0015396974032\\
0.855	5.0046313975165\\
0.856	5.00772146728613\\
0.857	5.01080990954085\\
0.858	5.01389672710116\\
0.859	5.01698192277937\\
0.86	5.0200654993796\\
0.861	5.02314745969785\\
0.862	5.02622780652201\\
0.863	5.02930654263192\\
0.864	5.03238367079935\\
0.865	5.03545919378808\\
0.866	5.03853311435392\\
0.867	5.04160543524475\\
0.868	5.04467615920051\\
0.869	5.0477452889533\\
0.87	5.05081282722735\\
0.871	5.05387877673908\\
0.872	5.05694314019713\\
0.873	5.06000592030239\\
0.874	5.06306711974805\\
0.875	5.06612674121956\\
0.876	5.06918478739479\\
0.877	5.07224126094389\\
0.878	5.07529616452948\\
0.879	5.07834950080659\\
0.88	5.08140127242272\\
0.881	5.08445148201785\\
0.882	5.08750013222449\\
0.883	5.09054722566772\\
0.884	5.09359276496517\\
0.885	5.0966367527271\\
0.886	5.09967919155641\\
0.887	5.10272008404868\\
0.888	5.10575943279215\\
0.889	5.10879724036783\\
0.89	5.11183350934947\\
0.891	5.11486824230359\\
0.892	5.11790144178953\\
0.893	5.12093311035952\\
0.894	5.12396325055858\\
0.895	5.12699186492465\\
0.896	5.13001895598865\\
0.897	5.13304452627437\\
0.898	5.13606857829863\\
0.899	5.13909111457125\\
0.9	5.1421121375951\\
0.901	5.14513164986604\\
0.902	5.14814965387311\\
0.903	5.1511661520984\\
0.904	5.15418114701718\\
0.905	5.15719464109787\\
0.906	5.16020663680209\\
0.907	5.16321713658466\\
0.908	5.16622614289368\\
0.909	5.16923365817051\\
0.91	5.17223968484981\\
0.911	5.17524422535956\\
0.912	5.17824728212107\\
0.913	5.18124885754908\\
0.914	5.18424895405168\\
0.915	5.1872475740304\\
0.916	5.19024471988023\\
0.917	5.19324039398963\\
0.918	5.19623459874057\\
0.919	5.19922733650854\\
0.92	5.20221860966257\\
0.921	5.20520842056528\\
0.922	5.20819677157289\\
0.923	5.21118366503523\\
0.924	5.21416910329579\\
0.925	5.21715308869172\\
0.926	5.22013562355389\\
0.927	5.22311671020686\\
0.928	5.22609635096895\\
0.929	5.22907454815225\\
0.93	5.23205130406262\\
0.931	5.23502662099978\\
0.932	5.23800050125721\\
0.933	5.24097294712233\\
0.934	5.24394396087639\\
0.935	5.24691354479458\\
0.936	5.24988170114599\\
0.937	5.25284843219369\\
0.938	5.25581374019469\\
0.939	5.25877762740001\\
0.94	5.26174009605474\\
0.941	5.26470114839792\\
0.942	5.2676607866627\\
0.943	5.27061901307634\\
0.944	5.27357582986017\\
0.945	5.27653123922967\\
0.946	5.27948524339442\\
0.947	5.28243784455827\\
0.948	5.28538904491916\\
0.949	5.28833884666934\\
0.95	5.29128725199521\\
0.951	5.29423426307748\\
0.952	5.29717988209112\\
0.953	5.30012411120542\\
0.954	5.30306695258398\\
0.955	5.30600840838472\\
0.956	5.30894848075996\\
0.957	5.31188717185637\\
0.958	5.31482448381504\\
0.959	5.3177604187715\\
0.96	5.32069497885568\\
0.961	5.32362816619203\\
0.962	5.32655998289943\\
0.963	5.3294904310913\\
0.964	5.33241951287559\\
0.965	5.33534723035474\\
0.966	5.33827358562584\\
0.967	5.3411985807805\\
0.968	5.34412221790492\\
0.969	5.34704449908001\\
0.97	5.34996542638124\\
0.971	5.35288500187877\\
0.972	5.35580322763744\\
0.973	5.3587201057168\\
0.974	5.36163563817109\\
0.975	5.36454982704932\\
0.976	5.36746267439524\\
0.977	5.3703741822474\\
0.978	5.37328435263909\\
0.979	5.37619318759848\\
0.98	5.37910068914852\\
0.981	5.38200685930704\\
0.982	5.38491170008673\\
0.983	5.38781521349514\\
0.984	5.39071740153479\\
0.985	5.39361826620306\\
0.986	5.39651780949229\\
0.987	5.39941603338979\\
0.988	5.40231293987785\\
0.989	5.40520853093373\\
0.99	5.40810280852971\\
0.991	5.41099577463312\\
0.992	5.41388743120633\\
0.993	5.41677778020675\\
0.994	5.41966682358691\\
0.995	5.4225545632944\\
0.996	5.42544100127197\\
0.997	5.42832613945747\\
0.998	5.4312099797839\\
0.999	5.43409252417946\\
1	5.43697377456752\\
1.001	5.43985373286662\\
1.002	5.44273240099056\\
1.003	5.44560978084837\\
1.004	5.44848587434431\\
1.005	5.45136068337793\\
1.006	5.45423420984405\\
1.007	5.4571064556328\\
1.008	5.45997742262961\\
1.009	5.46284711271528\\
1.01	5.46571552776593\\
1.011	5.46858266965304\\
1.012	5.4714485402435\\
1.013	5.47431314139959\\
1.014	5.47717647497899\\
1.015	5.48003854283481\\
1.016	5.48289934681561\\
1.017	5.48575888876544\\
1.018	5.48861717052379\\
1.019	5.49147419392565\\
1.02	5.49432996080149\\
1.021	5.49718447297738\\
1.022	5.50003773227487\\
1.023	5.50288974051106\\
1.024	5.50574049949864\\
1.025	5.5085900110459\\
1.026	5.51143827695668\\
1.027	5.51428529903048\\
1.028	5.51713107906239\\
1.029	5.5199756188432\\
1.03	5.5228189201593\\
1.031	5.52566098479276\\
1.032	5.5285018145214\\
1.033	5.53134141111866\\
1.034	5.53417977635375\\
1.035	5.53701691199159\\
1.036	5.53985281979286\\
1.037	5.542687501514\\
1.038	5.5455209589072\\
1.039	5.54835319372046\\
1.04	5.5511842076976\\
1.041	5.55401400257821\\
1.042	5.55684258009776\\
1.043	5.55966994198754\\
1.044	5.5624960899747\\
1.045	5.56532102578227\\
1.046	5.56814475112917\\
1.047	5.57096726773019\\
1.048	5.57378857729609\\
1.049	5.5766086815335\\
1.05	5.57942758214504\\
1.051	5.58224528082926\\
1.052	5.58506177928066\\
1.053	5.58787707918977\\
1.054	5.59069118224308\\
1.055	5.59350409012309\\
1.056	5.59631580450834\\
1.057	5.5991263270734\\
1.058	5.60193565948887\\
1.059	5.60474380342142\\
1.06	5.60755076053383\\
1.061	5.61035653248489\\
1.062	5.61316112092958\\
1.063	5.6159645275189\\
1.064	5.61876675390007\\
1.065	5.62156780171637\\
1.066	5.62436767260728\\
1.067	5.62716636820841\\
1.068	5.62996389015159\\
1.069	5.63276024006479\\
1.07	5.63555541957221\\
1.071	5.63834943029426\\
1.072	5.64114227384756\\
1.073	5.64393395184499\\
1.074	5.64672446589567\\
1.075	5.64951381760498\\
1.076	5.65230200857459\\
1.077	5.65508904040245\\
1.078	5.65787491468278\\
1.079	5.66065963300615\\
1.08	5.66344319695945\\
1.081	5.66622560812588\\
1.082	5.66900686808501\\
1.083	5.67178697841275\\
1.084	5.6745659406814\\
1.085	5.67734375645962\\
1.086	5.68012042731248\\
1.087	5.68289595480146\\
1.088	5.68567034048444\\
1.089	5.68844358591573\\
1.09	5.6912156926461\\
1.091	5.69398666222276\\
1.092	5.69675649618936\\
1.093	5.69952519608606\\
1.094	5.70229276344947\\
1.095	5.70505919981274\\
1.096	5.70782450670551\\
1.097	5.71058868565389\\
1.098	5.7133517381806\\
1.099	5.71611366580485\\
1.1	5.71887447004242\\
1.101	5.72163415240564\\
1.102	5.72439271440343\\
1.103	5.72715015754128\\
1.104	5.72990648332129\\
1.105	5.73266169324216\\
1.106	5.73541578879917\\
1.107	5.73816877148432\\
1.108	5.74092064278614\\
1.109	5.74367140418991\\
1.11	5.74642105717747\\
1.111	5.74916960322741\\
1.112	5.75191704381499\\
1.113	5.7546633804121\\
1.114	5.7574086144874\\
1.115	5.7601527475062\\
1.116	5.76289578093061\\
1.117	5.76563771621942\\
1.118	5.76837855482816\\
1.119	5.77111829820912\\
1.12	5.77385694781137\\
1.121	5.77659450508074\\
1.122	5.77933097145984\\
1.123	5.78206634838808\\
1.124	5.78480063730167\\
1.125	5.78753383963363\\
1.126	5.7902659568138\\
1.127	5.79299699026888\\
1.128	5.79572694142236\\
1.129	5.79845581169464\\
1.13	5.80118360250294\\
1.131	5.80391031526137\\
1.132	5.80663595138093\\
1.133	5.80936051226948\\
1.134	5.81208399933182\\
1.135	5.81480641396964\\
1.136	5.81752775758154\\
1.137	5.82024803156307\\
1.138	5.82296723730671\\
1.139	5.82568537620187\\
1.14	5.82840244963495\\
1.141	5.83111845898931\\
1.142	5.83383340564525\\
1.143	5.83654729098012\\
1.144	5.8392601163682\\
1.145	5.8419718831808\\
1.146	5.84468259278625\\
1.147	5.84739224654992\\
1.148	5.85010084583415\\
1.149	5.85280839199838\\
1.15	5.85551488639909\\
1.151	5.85822033038976\\
1.152	5.86092472532104\\
1.153	5.86362807254055\\
1.154	5.86633037339306\\
1.155	5.86903162922043\\
1.156	5.8717318413616\\
1.157	5.87443101115262\\
1.158	5.87712913992669\\
1.159	5.8798262290141\\
1.16	5.8825222797423\\
1.161	5.8852172934359\\
1.162	5.88791127141663\\
1.163	5.89060421500341\\
1.164	5.89329612551231\\
1.165	5.89598700425661\\
1.166	5.89867685254674\\
1.167	5.90136567169036\\
1.168	5.90405346299232\\
1.169	5.90674022775469\\
1.17	5.90942596727676\\
1.171	5.91211068285504\\
1.172	5.91479437578329\\
1.173	5.91747704735253\\
1.174	5.920158698851\\
1.175	5.92283933156423\\
1.176	5.92551894677503\\
1.177	5.92819754576347\\
1.178	5.93087512980689\\
1.179	5.93355170017996\\
1.18	5.93622725815465\\
1.181	5.93890180500023\\
1.182	5.9415753419833\\
1.183	5.94424787036775\\
1.184	5.94691939141487\\
1.185	5.94958990638324\\
1.186	5.95225941652882\\
1.187	5.95492792310492\\
1.188	5.95759542736221\\
1.189	5.96026193054874\\
1.19	5.96292743390995\\
1.191	5.96559193868867\\
1.192	5.96825544612511\\
1.193	5.97091795745689\\
1.194	5.97357947391905\\
1.195	5.97623999674406\\
1.196	5.9788995271618\\
1.197	5.98155806639961\\
1.198	5.98421561568223\\
1.199	5.98687217623189\\
1.2	5.98952774926825\\
1.201	5.99218233600848\\
1.202	5.99483593766715\\
1.203	5.99748855545639\\
1.204	6.00014019058574\\
1.205	6.00279084426232\\
1.206	6.00544051769067\\
1.207	6.0080892120729\\
1.208	6.01073692860859\\
1.209	6.01338366849486\\
1.21	6.01602943292639\\
1.211	6.01867422309537\\
1.212	6.0213180401915\\
1.213	6.02396088540213\\
1.214	6.02660275991206\\
1.215	6.02924366490372\\
1.216	6.03188360155711\\
1.217	6.03452257104981\\
1.218	6.03716057455693\\
1.219	6.03979761325126\\
1.22	6.04243368830314\\
1.221	6.04506880088051\\
1.222	6.04770295214896\\
1.223	6.05033614327168\\
1.224	6.0529683754095\\
1.225	6.05559964972087\\
1.226	6.05822996736189\\
1.227	6.06085932948628\\
1.228	6.0634877372455\\
1.229	6.06611519178859\\
1.23	6.06874169426225\\
1.231	6.07136724581093\\
1.232	6.07399184757671\\
1.233	6.07661550069935\\
1.234	6.07923820631635\\
1.235	6.08185996556286\\
1.236	6.08448077957176\\
1.237	6.08710064947368\\
1.238	6.08971957639691\\
1.239	6.0923375614675\\
1.24	6.09495460580922\\
1.241	6.0975707105436\\
1.242	6.10018587678991\\
1.243	6.10280010566515\\
1.244	6.10541339828411\\
1.245	6.10802575575934\\
1.246	6.11063717920113\\
1.247	6.11324766971758\\
1.248	6.11585722841457\\
1.249	6.11846585639577\\
1.25	6.12107355476262\\
1.251	6.12368032461442\\
1.252	6.12628616704823\\
1.253	6.12889108315892\\
1.254	6.13149507403922\\
1.255	6.13409814077968\\
1.256	6.13670028446865\\
1.257	6.13930150619233\\
1.258	6.1419018070348\\
1.259	6.14450118807795\\
1.26	6.14709965040154\\
1.261	6.14969719508323\\
1.262	6.15229382319846\\
1.263	6.15488953582065\\
1.264	6.15748433402103\\
1.265	6.16007821886873\\
1.266	6.1626711914308\\
1.267	6.16526325277215\\
1.268	6.16785440395563\\
1.269	6.17044464604198\\
1.27	6.17303398008984\\
1.271	6.17562240715581\\
1.272	6.17820992829439\\
1.273	6.18079654455804\\
1.274	6.18338225699709\\
1.275	6.1859670666599\\
1.276	6.18855097459275\\
1.277	6.19113398183984\\
1.278	6.19371608944338\\
1.279	6.1962972984435\\
1.28	6.19887760987837\\
1.281	6.20145702478406\\
1.282	6.20403554419465\\
1.283	6.20661316914226\\
1.284	6.20918990065693\\
1.285	6.21176573976674\\
1.286	6.21434068749778\\
1.287	6.2169147448741\\
1.288	6.21948791291782\\
1.289	6.22206019264909\\
1.29	6.22463158508603\\
1.291	6.22720209124484\\
1.292	6.22977171213973\\
1.293	6.23234044878296\\
1.294	6.23490830218486\\
1.295	6.23747527335377\\
1.296	6.24004136329612\\
1.297	6.24260657301641\\
1.298	6.24517090351717\\
1.299	6.24773435579905\\
1.3	6.25029693086076\\
1.301	6.25285862969907\\
1.302	6.25541945330889\\
1.303	6.25797940268318\\
1.304	6.26053847881303\\
1.305	6.26309668268762\\
1.306	6.26565401529421\\
1.307	6.26821047761825\\
1.308	6.27076607064326\\
1.309	6.27332079535088\\
1.31	6.2758746527209\\
1.311	6.27842764373124\\
1.312	6.28097976935794\\
1.313	6.28353103057524\\
1.314	6.28608142835545\\
1.315	6.28863096366911\\
1.316	6.29117963748487\\
1.317	6.29372745076959\\
1.318	6.29627440448824\\
1.319	6.29882049960401\\
1.32	6.30136573707826\\
1.321	6.30391011787052\\
1.322	6.30645364293854\\
1.323	6.30899631323823\\
1.324	6.3115381297237\\
1.325	6.31407909334729\\
1.326	6.31661920505954\\
1.327	6.31915846580918\\
1.328	6.32169687654317\\
1.329	6.32423443820671\\
1.33	6.32677115174321\\
1.331	6.32930701809431\\
1.332	6.33184203819989\\
1.333	6.33437621299806\\
1.334	6.33690954342519\\
1.335	6.33944203041592\\
1.336	6.34197367490309\\
1.337	6.34450447781784\\
1.338	6.34703444008957\\
1.339	6.34956356264592\\
1.34	6.35209184641285\\
1.341	6.35461929231455\\
1.342	6.35714590127353\\
1.343	6.35967167421055\\
1.344	6.36219661204469\\
1.345	6.36472071569332\\
1.346	6.3672439860721\\
1.347	6.369766424095\\
1.348	6.37228803067427\\
1.349	6.37480880672055\\
1.35	6.37732875314271\\
1.351	6.37984787084798\\
1.352	6.38236616074192\\
1.353	6.38488362372841\\
1.354	6.38740026070968\\
1.355	6.38991607258629\\
1.356	6.3924310602571\\
1.357	6.39494522461941\\
1.358	6.39745856656877\\
1.359	6.39997108699916\\
1.36	6.4024827868029\\
1.361	6.40499366687065\\
1.362	6.40750372809148\\
1.363	6.41001297135281\\
1.364	6.4125213975404\\
1.365	6.41502900753846\\
1.366	6.41753580222954\\
1.367	6.42004178249461\\
1.368	6.42254694921299\\
1.369	6.42505130326243\\
1.37	6.42755484551909\\
1.371	6.43005757685751\\
1.372	6.43255949815062\\
1.373	6.43506061026984\\
1.374	6.43756091408494\\
1.375	6.44006041046413\\
1.376	6.44255910027407\\
1.377	6.4450569843798\\
1.378	6.44755406364484\\
1.379	6.45005033893112\\
1.38	6.45254581109903\\
1.381	6.45504048100739\\
1.382	6.45753434951349\\
1.383	6.46002741747305\\
1.384	6.46251968574025\\
1.385	6.46501115516777\\
1.386	6.46750182660669\\
1.387	6.46999170090662\\
1.388	6.4724807789156\\
1.389	6.47496906148017\\
1.39	6.47745654944534\\
1.391	6.47994324365462\\
1.392	6.48242914494999\\
1.393	6.48491425417193\\
1.394	6.48739857215941\\
1.395	6.48988209974992\\
1.396	6.49236483777942\\
1.397	6.4948467870824\\
1.398	6.49732794849186\\
1.399	6.49980832283932\\
1.4	6.50228791095479\\
1.401	6.50476671366684\\
1.402	6.50724473180252\\
1.403	6.50972196618746\\
1.404	6.5121984176458\\
1.405	6.5146740870002\\
1.406	6.51714897507189\\
1.407	6.51962308268061\\
1.408	6.52209641064469\\
1.409	6.52456895978099\\
1.41	6.5270407309049\\
1.411	6.52951172483039\\
1.412	6.53198194237005\\
1.413	6.53445138433491\\
1.414	6.53692005153469\\
1.415	6.53938794477759\\
1.416	6.54185506487043\\
1.417	6.54432141261866\\
1.418	6.54678698882619\\
1.419	6.54925179429564\\
1.42	6.55171582982814\\
1.421	6.55417909622344\\
1.422	6.5566415942799\\
1.423	6.55910332479444\\
1.424	6.56156428856266\\
1.425	6.56402448637866\\
1.426	6.56648391903526\\
1.427	6.56894258732382\\
1.428	6.57140049203434\\
1.429	6.57385763395544\\
1.43	6.57631401387439\\
1.431	6.57876963257704\\
1.432	6.58122449084789\\
1.433	6.58367858947012\\
1.434	6.58613192922546\\
1.435	6.58858451089435\\
1.436	6.59103633525586\\
1.437	6.59348740308768\\
1.438	6.59593771516619\\
1.439	6.59838727226638\\
1.44	6.60083607516196\\
1.441	6.60328412462523\\
1.442	6.60573142142722\\
1.443	6.60817796633756\\
1.444	6.6106237601246\\
1.445	6.61306880355536\\
1.446	6.61551309739551\\
1.447	6.61795664240943\\
1.448	6.62039943936015\\
1.449	6.62284148900942\\
1.45	6.62528279211767\\
1.451	6.627723349444\\
1.452	6.63016316174622\\
1.453	6.63260222978089\\
1.454	6.63504055430315\\
1.455	6.637478136067\\
1.456	6.63991497582501\\
1.457	6.64235107432854\\
1.458	6.64478643232765\\
1.459	6.6472210505711\\
1.46	6.6496549298064\\
1.461	6.65208807077975\\
1.462	6.6545204742361\\
1.463	6.65695214091913\\
1.464	6.65938307157124\\
1.465	6.66181326693358\\
1.466	6.66424272774604\\
1.467	6.6666714547472\\
1.468	6.66909944867449\\
1.469	6.67152671026398\\
1.47	6.67395324025055\\
1.471	6.67637903936784\\
1.472	6.6788041083482\\
1.473	6.68122844792279\\
1.474	6.6836520588215\\
1.475	6.68607494177301\\
1.476	6.68849709750474\\
1.477	6.69091852674291\\
1.478	6.69333923021248\\
1.479	6.69575920863723\\
1.48	6.6981784627397\\
1.481	6.70059699324121\\
1.482	6.70301480086185\\
1.483	6.70543188632054\\
1.484	6.70784825033495\\
1.485	6.71026389362159\\
1.486	6.71267881689572\\
1.487	6.71509302087142\\
1.488	6.71750650626158\\
1.489	6.7199192737779\\
1.49	6.72233132413086\\
1.491	6.72474265802977\\
1.492	6.72715327618279\\
1.493	6.72956317929682\\
1.494	6.73197236807764\\
1.495	6.73438084322983\\
1.496	6.7367886054568\\
1.497	6.73919565546078\\
1.498	6.74160199394287\\
1.499	6.74400762160293\\
1.5	6.74641253913971\\
1.501	6.74881674725079\\
1.502	6.7512202466326\\
1.503	6.7536230379804\\
1.504	6.75602512198829\\
1.505	6.75842649934923\\
1.506	6.76082717075504\\
1.507	6.76322713689637\\
1.508	6.76562639846276\\
1.509	6.7680249561426\\
1.51	6.77042281062313\\
1.511	6.77281996259046\\
1.512	6.77521641272958\\
1.513	6.77761216172434\\
1.514	6.78000721025746\\
1.515	6.78240155901055\\
1.516	6.78479520866409\\
1.517	6.78718815989745\\
1.518	6.78958041338886\\
1.519	6.79197196981547\\
1.52	6.79436282985328\\
1.521	6.79675299417723\\
1.522	6.7991424634611\\
1.523	6.80153123837761\\
1.524	6.80391931959836\\
1.525	6.80630670779387\\
1.526	6.80869340363352\\
1.527	6.81107940778563\\
1.528	6.81346472091745\\
1.529	6.8158493436951\\
1.53	6.81823327678363\\
1.531	6.82061652084701\\
1.532	6.82299907654813\\
1.533	6.82538094454878\\
1.534	6.82776212550972\\
1.535	6.83014262009059\\
1.536	6.83252242894999\\
1.537	6.83490155274541\\
1.538	6.83727999213335\\
1.539	6.83965774776915\\
1.54	6.84203482030717\\
1.541	6.84441121040067\\
1.542	6.84678691870187\\
1.543	6.84916194586192\\
1.544	6.85153629253092\\
1.545	6.85390995935793\\
1.546	6.85628294699099\\
1.547	6.85865525607701\\
1.548	6.86102688726195\\
1.549	6.86339784119069\\
1.55	6.86576811850708\\
1.551	6.86813771985391\\
1.552	6.87050664587298\\
1.553	6.87287489720502\\
1.554	6.87524247448977\\
1.555	6.8776093783659\\
1.556	6.87997560947109\\
1.557	6.88234116844199\\
1.558	6.88470605591421\\
1.559	6.88707027252239\\
1.56	6.88943381890012\\
1.561	6.89179669567998\\
1.562	6.89415890349355\\
1.563	6.89652044297139\\
1.564	6.89888131474307\\
1.565	6.90124151943714\\
1.566	6.90360105768118\\
1.567	6.90595993010172\\
1.568	6.90831813732434\\
1.569	6.91067567997362\\
1.57	6.91303255867313\\
1.571	6.91538877404544\\
1.572	6.91774432671215\\
1.573	6.92009921729391\\
1.574	6.92245344641031\\
1.575	6.92480701468003\\
1.576	6.92715992272071\\
1.577	6.92951217114908\\
1.578	6.93186376058083\\
1.579	6.93421469163073\\
1.58	6.93656496491253\\
1.581	6.93891458103907\\
1.582	6.94126354062217\\
1.583	6.94361184427274\\
1.584	6.94595949260066\\
1.585	6.9483064862149\\
1.586	6.95065282572348\\
1.587	6.95299851173342\\
1.588	6.95534354485082\\
1.589	6.95768792568084\\
1.59	6.96003165482764\\
1.591	6.9623747328945\\
1.592	6.96471716048367\\
1.593	6.96705893819656\\
1.594	6.96940006663356\\
1.595	6.97174054639414\\
1.596	6.97408037807688\\
1.597	6.97641956227935\\
1.598	6.97875809959824\\
1.599	6.98109599062929\\
1.6	6.9834332359673\\
1.601	6.9857698362062\\
1.602	6.98810579193891\\
1.603	6.9904411037575\\
1.604	6.99277577225306\\
1.605	6.99510979801584\\
1.606	6.9974431816351\\
1.607	6.99977592369922\\
1.608	7.00210802479565\\
1.609	7.00443948551096\\
1.61	7.00677030643077\\
1.611	7.00910048813981\\
1.612	7.01143003122195\\
1.613	7.01375893626007\\
1.614	7.01608720383624\\
1.615	7.01841483453157\\
1.616	7.02074182892628\\
1.617	7.02306818759974\\
1.618	7.02539391113036\\
1.619	7.02771900009571\\
1.62	7.03004345507246\\
1.621	7.03236727663638\\
1.622	7.03469046536237\\
1.623	7.03701302182444\\
1.624	7.0393349465957\\
1.625	7.04165624024842\\
1.626	7.04397690335396\\
1.627	7.04629693648284\\
1.628	7.04861634020465\\
1.629	7.05093511508815\\
1.63	7.05325326170123\\
1.631	7.05557078061091\\
1.632	7.05788767238331\\
1.633	7.06020393758373\\
1.634	7.06251957677659\\
1.635	7.06483459052542\\
1.636	7.06714897939295\\
1.637	7.06946274394102\\
1.638	7.07177588473061\\
1.639	7.07408840232186\\
1.64	7.07640029727404\\
1.641	7.07871157014558\\
1.642	7.08102222149409\\
1.643	7.0833322518763\\
1.644	7.08564166184809\\
1.645	7.0879504519645\\
1.646	7.09025862277979\\
1.647	7.09256617484728\\
1.648	7.09487310871953\\
1.649	7.09717942494824\\
1.65	7.09948512408427\\
1.651	7.10179020667764\\
1.652	7.10409467327758\\
1.653	7.10639852443246\\
1.654	7.10870176068982\\
1.655	7.11100438259638\\
1.656	7.11330639069805\\
1.657	7.11560778553991\\
1.658	7.11790856766621\\
1.659	7.12020873762039\\
1.66	7.1225082959451\\
1.661	7.12480724318212\\
1.662	7.12710557987246\\
1.663	7.12940330655632\\
1.664	7.13170042377307\\
1.665	7.13399693206127\\
1.666	7.13629283195869\\
1.667	7.13858812400229\\
1.668	7.14088280872824\\
1.669	7.14317688667187\\
1.67	7.14547035836776\\
1.671	7.14776322434966\\
1.672	7.15005548515054\\
1.673	7.15234714130256\\
1.674	7.15463819333711\\
1.675	7.15692864178475\\
1.676	7.15921848717533\\
1.677	7.1615077300378\\
1.678	7.16379637090042\\
1.679	7.16608441029062\\
1.68	7.16837184873505\\
1.681	7.17065868675959\\
1.682	7.17294492488934\\
1.683	7.1752305636486\\
1.684	7.17751560356092\\
1.685	7.17980004514909\\
1.686	7.18208388893508\\
1.687	7.18436713544012\\
1.688	7.18664978518466\\
1.689	7.18893183868839\\
1.69	7.19121329647022\\
1.691	7.19349415904831\\
1.692	7.19577442694005\\
1.693	7.19805410066207\\
1.694	7.20033318073023\\
1.695	7.20261166765962\\
1.696	7.20488956196465\\
1.697	7.20716686415886\\
1.698	7.20944357475512\\
1.699	7.2117196942655\\
1.7	7.21399522320136\\
1.701	7.21627016207327\\
1.702	7.2185445113911\\
1.703	7.22081827166391\\
1.704	7.22309144340007\\
1.705	7.22536402710717\\
1.706	7.22763602329211\\
1.707	7.22990743246098\\
1.708	7.23217825511919\\
1.709	7.23444849177137\\
1.71	7.23671814292145\\
1.711	7.23898720907258\\
1.712	7.24125569072724\\
1.713	7.24352358838714\\
1.714	7.24579090255326\\
1.715	7.24805763372586\\
1.716	7.25032378240444\\
1.717	7.25258934908786\\
1.718	7.25485433427416\\
1.719	7.25711873846073\\
1.72	7.25938256214417\\
1.721	7.26164580582044\\
1.722	7.26390846998471\\
1.723	7.26617055513147\\
1.724	7.26843206175452\\
1.725	7.27069299034688\\
1.726	7.27295334140091\\
1.727	7.27521311540825\\
1.728	7.27747231285981\\
1.729	7.2797309342458\\
1.73	7.28198898005576\\
1.731	7.28424645077848\\
1.732	7.28650334690205\\
1.733	7.28875966891388\\
1.734	7.29101541730068\\
1.735	7.29327059254843\\
1.736	7.29552519514244\\
1.737	7.29777922556732\\
1.738	7.30003268430697\\
1.739	7.30228557184461\\
1.74	7.30453788866278\\
1.741	7.3067896352433\\
1.742	7.30904081206731\\
1.743	7.31129141961527\\
1.744	7.31354145836696\\
1.745	7.31579092880145\\
1.746	7.31803983139715\\
1.747	7.32028816663177\\
1.748	7.32253593498235\\
1.749	7.32478313692524\\
1.75	7.32702977293613\\
1.751	7.32927584349002\\
1.752	7.33152134906123\\
1.753	7.33376629012341\\
1.754	7.33601066714953\\
1.755	7.33825448061191\\
1.756	7.34049773098216\\
1.757	7.34274041873129\\
1.758	7.34498254432958\\
1.759	7.34722410824663\\
1.76	7.34946511095145\\
1.761	7.35170555291232\\
1.762	7.35394543459688\\
1.763	7.35618475647212\\
1.764	7.35842351900436\\
1.765	7.36066172265925\\
1.766	7.36289936790181\\
1.767	7.36513645519638\\
1.768	7.36737298500664\\
1.769	7.36960895779564\\
1.77	7.37184437402577\\
1.771	7.37407923415876\\
1.772	7.37631353865573\\
1.773	7.37854728797708\\
1.774	7.3807804825826\\
1.775	7.38301312293147\\
1.776	7.38524520948216\\
1.777	7.38747674269256\\
1.778	7.38970772301987\\
1.779	7.39193815092065\\
1.78	7.39416802685086\\
1.781	7.39639735126581\\
1.782	7.39862612462014\\
1.783	7.4008543473679\\
1.784	7.40308201996245\\
1.785	7.40530914285657\\
1.786	7.4075357165024\\
1.787	7.4097617413514\\
1.788	7.41198721785447\\
1.789	7.41421214646185\\
1.79	7.41643652762313\\
1.791	7.41866036178732\\
1.792	7.42088364940278\\
1.793	7.42310639091723\\
1.794	7.4253285867778\\
1.795	7.42755023743099\\
1.796	7.42977134332267\\
1.797	7.4319919048981\\
1.798	7.43421192260192\\
1.799	7.43643139687817\\
1.8	7.43865032817025\\
1.801	7.44086871692095\\
1.802	7.44308656357245\\
1.803	7.44530386856636\\
1.804	7.44752063234359\\
1.805	7.44973685534453\\
1.806	7.45195253800891\\
1.807	7.45416768077589\\
1.808	7.45638228408399\\
1.809	7.45859634837114\\
1.81	7.46080987407467\\
1.811	7.4630228616313\\
1.812	7.46523531147717\\
1.813	7.46744722404779\\
1.814	7.46965859977809\\
1.815	7.4718694391024\\
1.816	7.47407974245446\\
1.817	7.47628951026741\\
1.818	7.47849874297378\\
1.819	7.48070744100555\\
1.82	7.48291560479404\\
1.821	7.48512323477006\\
1.822	7.48733033136379\\
1.823	7.48953689500478\\
1.824	7.49174292612209\\
1.825	7.4939484251441\\
1.826	7.49615339249866\\
1.827	7.49835782861303\\
1.828	7.50056173391387\\
1.829	7.50276510882728\\
1.83	7.50496795377875\\
1.831	7.50717026919322\\
1.832	7.50937205549504\\
1.833	7.51157331310798\\
1.834	7.51377404245524\\
1.835	7.51597424395945\\
1.836	7.51817391804265\\
1.837	7.52037306512631\\
1.838	7.52257168563136\\
1.839	7.52476977997812\\
1.84	7.52696734858635\\
1.841	7.52916439187526\\
1.842	7.53136091026346\\
1.843	7.53355690416904\\
1.844	7.53575237400949\\
1.845	7.53794732020174\\
1.846	7.54014174316217\\
1.847	7.54233564330659\\
1.848	7.54452902105024\\
1.849	7.54672187680783\\
1.85	7.54891421099346\\
1.851	7.55110602402074\\
1.852	7.55329731630266\\
1.853	7.55548808825169\\
1.854	7.55767834027976\\
1.855	7.5598680727982\\
1.856	7.56205728621782\\
1.857	7.56424598094885\\
1.858	7.56643415740103\\
1.859	7.56862181598349\\
1.86	7.57080895710482\\
1.861	7.57299558117311\\
1.862	7.57518168859584\\
1.863	7.57736727977998\\
1.864	7.57955235513196\\
1.865	7.58173691505765\\
1.866	7.58392095996241\\
1.867	7.586104490251\\
1.868	7.58828750632769\\
1.869	7.59047000859618\\
1.87	7.59265199745968\\
1.871	7.59483347332082\\
1.872	7.59701443658168\\
1.873	7.59919488764386\\
1.874	7.60137482690838\\
1.875	7.60355425477575\\
1.876	7.60573317164595\\
1.877	7.60791157791839\\
1.878	7.610089473992\\
1.879	7.61226686026516\\
1.88	7.61444373713572\\
1.881	7.61662010500101\\
1.882	7.61879596425783\\
1.883	7.62097131530244\\
1.884	7.62314615853061\\
1.885	7.62532049433756\\
1.886	7.62749432311799\\
1.887	7.62966764526608\\
1.888	7.63184046117551\\
1.889	7.63401277123941\\
1.89	7.63618457585042\\
1.891	7.63835587540063\\
1.892	7.64052667028165\\
1.893	7.64269696088455\\
1.894	7.64486674759989\\
1.895	7.64703603081771\\
1.896	7.64920481092756\\
1.897	7.65137308831846\\
1.898	7.65354086337894\\
1.899	7.65570813649696\\
1.9	7.65787490806007\\
1.901	7.6600411784552\\
1.902	7.66220694806889\\
1.903	7.66437221728707\\
1.904	7.66653698649522\\
1.905	7.66870125607831\\
1.906	7.67086502642079\\
1.907	7.67302829790664\\
1.908	7.6751910709193\\
1.909	7.67735334584174\\
1.91	7.6795151230564\\
1.911	7.68167640294526\\
1.912	7.68383718588976\\
1.913	7.68599747227088\\
1.914	7.68815726246907\\
1.915	7.69031655686434\\
1.916	7.69247535583614\\
1.917	7.69463365976345\\
1.918	7.69679146902477\\
1.919	7.69894878399811\\
1.92	7.701105605061\\
1.921	7.70326193259042\\
1.922	7.70541776696293\\
1.923	7.70757310855457\\
1.924	7.7097279577409\\
1.925	7.71188231489698\\
1.926	7.71403618039742\\
1.927	7.71618955461631\\
1.928	7.71834243792725\\
1.929	7.72049483070341\\
1.93	7.72264673331743\\
1.931	7.72479814614147\\
1.932	7.72694906954725\\
1.933	7.72909950390597\\
1.934	7.73124944958838\\
1.935	7.73339890696472\\
1.936	7.73554787640478\\
1.937	7.73769635827787\\
1.938	7.73984435295283\\
1.939	7.74199186079799\\
1.94	7.74413888218128\\
1.941	7.74628541747008\\
1.942	7.74843146703133\\
1.943	7.75057703123152\\
1.944	7.75272211043666\\
1.945	7.75486670501223\\
1.946	7.75701081532335\\
1.947	7.75915444173461\\
1.948	7.76129758461012\\
1.949	7.76344024431354\\
1.95	7.76558242120811\\
1.951	7.76772411565654\\
1.952	7.7698653280211\\
1.953	7.77200605866361\\
1.954	7.77414630794543\\
1.955	7.77628607622744\\
1.956	7.77842536387007\\
1.957	7.78056417123329\\
1.958	7.78270249867661\\
1.959	7.78484034655911\\
1.96	7.78697771523937\\
1.961	7.78911460507554\\
1.962	7.79125101642531\\
1.963	7.79338694964591\\
1.964	7.79552240509413\\
1.965	7.79765738312631\\
1.966	7.79979188409829\\
1.967	7.80192590836553\\
1.968	7.80405945628301\\
1.969	7.80619252820526\\
1.97	7.80832512448633\\
1.971	7.81045724547988\\
1.972	7.81258889153909\\
1.973	7.81472006301671\\
1.974	7.81685076026503\\
1.975	7.81898098363588\\
1.976	7.82111073348069\\
1.977	7.82324001015042\\
1.978	7.82536881399559\\
1.979	7.82749714536627\\
1.98	7.82962500461212\\
1.981	7.83175239208232\\
1.982	7.83387930812566\\
1.983	7.83600575309044\\
1.984	7.83813172732455\\
1.985	7.84025723117545\\
1.986	7.84238226499014\\
1.987	7.8445068291152\\
1.988	7.84663092389679\\
1.989	7.84875454968059\\
1.99	7.85087770681191\\
1.991	7.85300039563559\\
1.992	7.85512261649604\\
1.993	7.85724436973724\\
1.994	7.85936565570275\\
1.995	7.86148647473568\\
1.996	7.86360682717874\\
1.997	7.86572671337422\\
1.998	7.86784613366391\\
1.999	7.86996508838928\\
2	7.8720835778913\\
};
\end{axis}
\end{tikzpicture}%
\caption{Variation du prix du CALL en fonction date d'échéance T}
\label{fig:call_t}
\end{figure}
% section question_1 (end)  

On constante une fonction concave et croissante. C'est logique que le prix soit croissant avec la date d'échéance T. Si T est plus loin, plus les perspectives de gains sont élevées. La concavité est liée au fait que si la date d'écheance est grande, avoir un jour en plus/en moins ne change rien pour de grandes valeurs de T. 
