\section{TD 17}


\subsection{Rappels}


\paragraph{Approximation de Stirling} 
$$ ln(X!)\approx Xln(X) - X$$


\paragraph{Somme série géométrique}
$$ \sum_{i=p}^\infty r^n=\frac{r^p}{1-r}$$


\subsection{Notions importantes}


\paragraph{Ensembles canoniques et fonctions de partition}

\begin{itemize}[label=\ding{69}]

	\item Variation du nombre de molécules et de l'énergie implique ensemble grand canonique.

	\item On peut considérer que le réservoir n'a pas de modification de l'énergie ni du nombre de ses molécules

	\[
		E_R>>E_s \qquad N_R>>N_s
	\]

	C'est à dire qu'on peut le traiter comme un sytème isolé

	\item Pour calculer $\Xi$ on calcule initialement $\xi$ puis on utilise 

	\[
		\Xi=\xi^N
	\]

	Qui n'est valable que dans le cas de particules \textbf{discernables}

	\item Il est important de comprendre que les fonctions de partitions peuvent se "partitionner" en fonctions de partition de chaque terme énergétique indépendant:

	\begin{align*}
		E=E_1+E_2 && \Xi= \Xi_1 \Xi_2
	\end{align*}


	\item À l'équilibre 

	\begin{align*}
		\bar{N}= \frac{1}{\beta}\frac{\partial ln\Xi}{\partial \mu}
		&& \mu=-\frac{1}{\beta}\frac{\partial lnZ}{\partial Ng}=-\frac{1}{\beta}\frac{\partial ln\Xi}{\partial Ng}
	\end{align*}

\end{itemize}


\paragraph{Dans le cas d'un gaz parfait} 

\[
	\xi = V\left(\frac{m}{2 \pi \beta_R \hbar^2}\right)^{3/2}
\]

et comme le gaz fait office de réservoir et ses particules sons indiscernables

\[
	Z= \frac{\xi^N}{N!}=V\left(\frac{m}{2 \pi \beta_R \hbar^2}\right)^{3N/2}
\]

Et la fameuse loi : 
\[
	PV=NkT
\]


\subsection{Résultats}


La température permet comme il est logique de faciliter la libération des particules piégés en leur donnant plus d'énergie pour sortir du puits.
On obtiens les lois d'adsorption et de Clapeyron avec cette modélisation.