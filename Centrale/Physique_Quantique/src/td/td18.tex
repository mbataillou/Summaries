\section{TD 18}



\subsection{Rappels}


\paragraph{Taylor}

\[
	e^x\approx 1+x+\frac{x^2}{2}+...+\frac{x^n}{n!}
\]


\subsection{Notions importantes}


\paragraph{Nombre moyen d'occupation}

On vas réaliser quelques précisions sur ce que signifie en fait le nombre moyen d'occupation. En effet on a présenté antérieurement (dans le cas des bosons): 

\[
	\bar{N}=\sum_i \frac{1}{e^{-\beta (\epsilon_i-\mu)}-1}
\]

Ce qui est tout à fait juste mais cette expression ne prend pas en compte la multiplicité que peut avoir une énergie. En effet cette expression nous dit qu'à l'énergie $\epsilon_i$ il devrait y avoir $\bar{N}=\sum_i \frac{1}{e^{-\beta (\epsilon_i-\mu)}-1}$
Mais que se passe-t-il si cette énergie est dégénérée ? Et bien il y aura autant de fois plus de particules qu'elle est dégénérée c'est à dire :

\[
	\bar{N}=\sum_i \frac{g(\epsilon_i)}{e^{-\beta (\epsilon_i-\mu)}-1}
\]

L'explication est assez logique si je dis que dans une boite si par exemple je veux compter les éléments menant à une norme = 2 (norme euclidienne). J'ai dans $\mathbb{R}$ 2 possibilitées puis dans $\mathbb{R}^2$ 4 possibilités. C'est à dire qu'il y a plusieurs façons d'arriver à une norme=2. Le raisonnement est le même, il y a plusieurs formes d'arriver à une même énergie.

\paragraph{Le potentiel chimique}

Cette subsection est importante car on aura souvent besoin de le trouver. La méthode que l'on utilise dans ce TD est la générale mais elle peut s'avérer être complexe à mettre en oeuvre.

\begin{itemize}[label=\ding{69}]

	\item Première méthode (utilisée dans ce TD)

	\[
		\bar{N}=\sum_i \frac{g(\epsilon_i)}{e^{-\beta (\epsilon_i-\mu)}-1}
	\]

	Et on connait $\bar{N}$ il nous suffit donc d'isoler $\mu$.

	\item Deuxième méthode :

	Elle consiste à employer des développement limités si $\beta (\epsilon-\mu)$

	\item Troisième méthode (cas particulier): 

	Dans le cas des fermions on peut jouer avec la limite quand $T\rightarrow 0$ pour trouver $\mu_F$

\end{itemize}


\paragraph{Chaleur molaire}

Attention ! Si l'on a l'énergie moyenne d'une particule on aura :
\[
	 C=N_a \frac{\partial \bar{E}}{\partial T}
\]
Si on a l'énergie moyenne de l'ensemble:
\[
	 C= \frac{\partial \bar{E}}{\partial T}
\]


\paragraph{Les ordres de grandeur de l'énergie thermique}

\[
	E_t=\frac{fkT}{2}
\] 
Avec $f$ le nombre de degrés de liberté.

A température ambiante $E_t=1/40$ eV donc tout énergie de l'ordre de l'eV est très supérieure à l'énergie thermique.