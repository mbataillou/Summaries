\section{TD 14}



\subsection{Rappels}


\paragraph{L'énergie potentielle d'interaction entre 2 charges }

$$E= \frac{q_1q_2}{4\pi \epsilon_0 |r|}$$


\paragraph{L'approximation de Bohr-Oppenheimer}

Elle se base sur le fait que la masse du noyau est beaucoup plus élevée que celle de l'électron ($\approx 10^3$) ce qui nous permet de négliger l'énergie cinétique du noyau par rapport à celle de l'électron en se plaçant dans le référentiel du centre de masse.


\paragraph{L'atome d'hydrogène}

La fonction d'onde de l'état fondamental:
$$\phi_{1s}=\frac{1}{\sqrt{\pi}}\left(\frac{Z}{a}\right)^{3/2}e^{\frac{Zr}{a}}$$


\subsection{Notions Importantes}


\paragraph{Formulaire}

\begin{center}
	\fbox{
	\begin{tabular}{ccc}
		$S_{ij}=\int\phi_{1s}(r_i)\phi_{1s}(r_j)d^3r=<\phi_i|\phi_j> $ &  & $H_{ij}=\int\phi_{1s}(r_i)H\phi_{1s}(r_j)d^3r=<\phi_i|H|\phi_j>$ \\
		$ S_{11}=S_{22}$ & & $H_{11}=H_{22}= E_H+2E_H[\frac{a}{d}-(1+\frac{a}{d})e^{-d/a}]$ \\
		$S_{12}=S_{21}=S= (1+\frac{d}{a}+\frac{d^2}{3a^2})e^{-\frac{d}{a}}$ &  & $ H_{11}=H_{22}= E_H S_{12}+2E_H(1+\frac{d}{a})e^{-\frac{d}{a}} $\\
		$ a=\frac{\mu e^2}{4\pi\epsilon_0\hbar^2}= 52,9.10^{-12}m $ &  & $ E_H=\frac{ e^2}{8\pi\epsilon_0 a}= -13,6eV $
	\end{tabular}
	}
\end{center}


\paragraph{La méthode variationelle}

Elle se base sur le fait que si $\phi_t$ est un état quelconque qui s'écrit donc sur la base des états propres $\phi_t=\sum_i c_i |\varphi_i \rangle$ on a 
\[
	\langle \phi_t|\hat{H}|\phi_t \rangle \geq E_0 |c_i|^2=E_0\langle \phi_t|\phi_t \rangle
\]

On va donc chercher à minimiser la fonctionnelle 
\[
	E=\frac{\langle \phi_t|\hat{H}|\phi_t \rangle}{\langle \phi_t|\phi_t \rangle}
\]
De plus comme $\phi_t$ est une superposition linéaire d'états on utilisera la méthode de Ritz qui dit que minimiser la fonctionnelle revient à résoudre
\[
	Hc-ESc=0
\]
avec la condition que le déterminant $|H-ES|$ soit différent de 0 (pour éviter la solution nulle). Et la normalisation de la fonction d'onde nous donne les solutions de l'approximation.


\paragraph{La stabilité}

Un état est plus ou moins stable si son énergie de liaison (minimum de la fonctionnelle) est plus est plus ou moins élevé.\\
Un point important est le fait que s'il n'y a pas de minimum de l'énergie en fonction de d/a il ne peut pas y avoir de stabilité car la molécule va se dissocier pour minimiser son énergie, on parle d'état anti-liant. 


\subsection{Résultats}


La molécule $H_2^+$ est plus stable que l'atome H