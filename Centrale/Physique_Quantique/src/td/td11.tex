\section{TD 11}



\subsection{Rappels}


\paragraph{Développements de Taylor (en 0)}
\[
	(1+x)^a\approx 1+ax+\frac{a(a-1)}{2}x^2+...+\frac{a(a-1)...(a-n+1)}{n!}x^n+o(n)
\]
\[
	e^x\approx 1+x+\frac{x^2}{2}+...+\frac{x^n}{n!}+o(n)
\]

\paragraph{L'oscillateur harmonique}


\begin{itemize}[label=\ding{69}]

	\item Le hamiltonien et ses valeurs propres:

	\begin{align*}
		\hat{H}= \frac{p^2}{2m}+ \frac{1}{2}mw^2 &&
		E_n= \ \left(n+\frac{1}{2}\right), \; n\in \mathbb{N}
	\end{align*}

	\item Les opérateurs création et annihilation: 

	\begin{align*}
		a^{\dag}=\frac{1}{2}\left(\frac{x}{\sigma_{x_0}}-i\frac{p}{\sigma_{p_0}}\right) &&
		a=\frac{1}{2}\left(\frac{x}{\sigma_{x_0}}+i\frac{p}{\sigma_{p_0}}\right) &&
		N=a^{\dag}a
	\end{align*}

	\item Leurs actions sur un état $|n \rangle$ :

	\begin{align*}
		a^{\dag}|n \rangle = \sqrt{n+1}|n+1 \rangle &&
		a|n \rangle = \sqrt{n}|n-1 \rangle &&
		N|n \rangle = n|n \rangle
	\end{align*}

	\item L'opérateur $\hat{x}$:

	\begin{align*}
		x=\sigma_{x_0} (a^{\dag}+a)\qquad \quad &&
		x^2=\sigma_{x_0}^2 (a^2+ aa^{\dag}+a^{\dag}a+a^{\dag 2})=\sigma_{x_0}^2 (a^2+2N+1+a^{\dag 2}) \\
		[a,a^{\dag}]=aa^{\dag}-a^{\dag}a=1 && 
		x^3=x^2x=\sigma_{x_0}^2(a^2+2N+1+a^{\dag 2})(a+a^{\dag})
	\end{align*}

	\item L'action de $x, x^2, x^3$:

	\[
		\langle n|x^2|k \rangle=\sigma_{x_0}^2 (\sqrt{k-1}\sqrt{k}\delta_{n,k-2}+(1+2k)\delta_{n,k}+\sqrt{k+1}\sqrt{k+2}\delta_{n,k+2})
	\]

	\[
		\langle n|x^3|k \rangle=\sigma_{x_0}^3 (\sqrt{k-3}\sqrt{k-1}\sqrt{k}\delta_{n,k-2}+3k^{3/2}\delta_{n,k-1}$$ $$
		+3(k+1)^{3/2}\delta_{n,k+1}+\sqrt{k+1}\sqrt{k+2}\sqrt{k+3}\delta_{n,k+3})
	\]

\end{itemize}


\subsection{Notions importantes}


\paragraph{Analyse du potentiel}

\begin{itemize}

	\item La parité: 


	La parité du potentiel implique la nécessité de la parité ou imparité des fonctions propres. De plus si on le couple au théorème de Sturm-Liouville, à savoir, \textit{on peut classer les niveaux par valeur croissante de l'énergie en fonction du nombre de nœuds (changement de signe) de la fonction d'onde }. On obtiens que les états associés à une valeur paire de l'énergie son pairs et vice-versa.

	On tire de cette information que la fonction d'onde est nécessairement paire dans le cas d'un potentiel pair. 

	Ce qui implique:
	\[
		\int_{-\infty}^{+\infty}f(x)|\phi|^2=0 \qquad \text{avec f une fonction impaire}
	\]
	\item La forme: 

	Si le potentiel est plus raide (forme de la fonction) dans une zone que dans une autre on aura une diminution de la probabilité de présence dans cette zone par rapport à une autre ou serait plus "mou" (moins raide).

\end{itemize}


\subsection*{Les formules des perturbations}


\begin{itemize}[label=\ding{69}]

	\item Au premier ordre:

	\begin{itemize}[label=\ding{71}]

		\item L'énergie:

		\[
			E \approx E_i^{(0)} + \langle \varphi_i^{(0)} | \hat{W} |\varphi_i^{(0)} \rangle
		\]

		\item La fonction propre :

		\[
			| \varphi \rangle \approx
			| \varphi_i^{(0)} \rangle +
			\sum_{i\neq k}\frac{\langle \varphi_i^{(0)} | \hat{W} |\varphi_k^{(0)} \rangle}{E_i^{(0)}-E_k^{(0)}}
			| \varphi_k^{(0)}\rangle 
		\]

		\item La fonction d'onde:

		\[
			| \varphi \rangle \approx
			| \varphi_i^{(0)} \rangle +
			\sum_{i\neq k}\frac{|\langle \varphi_i^{(0)} | \hat{W} |\varphi_k^{(0)} \rangle|^2}{E_i^{(0)}-E_k^{(0)}}
			| \varphi_k^{(0)}\rangle
		\]

	\end{itemize}

	\item Au deuxième ordre:

	\[
		\approx E_i^{(0)} + \langle \varphi_i^{(0)} | \hat{W} |\varphi_i^{(0)} \rangle +\sum_{i\neq k}\frac{|\langle \varphi_k^{(0)} | \hat{W} |\varphi_i^{(0)} \rangle|^2}{E_i^{(0)}-E_k^{(0)}}
	\]

\end{itemize}


\subsection{Résultats}


\begin{itemize}[label=\ding{69}]

	\item On a vu que l'approche par la théorie des perturbations (quantique) et le développement limité arrivent au même résultat. Ce qui confirme la proximité du développement de l'approche quantique de ce dernier.

	\item Dans le cas du potentiel morse on observe une diminutiond e l'écart entre niveaux énergétiques lorsque n augmente.
	
\end{itemize}