\section{TD 19}



\subsection{Rappels}


\paragraph{Aire d'un polyèdre régulier}

\[
	A=\frac{nah}{2}
\]

Avec h le rayon du cercle inscrit  (apothème), a la longueur d'un côté et n le nombre de côtés.


\paragraph{Énergie puits infini 2D }

\[
	\frac{(n_x^2+n_y^2 )\pi^2\hbar^2}{2mL^2}
\]


\subsection{Notions importantes}


\paragraph{Les ordres de grandeur de l'énergie thermique}

\[
	E_t=\frac{fkT}{2}
\] Avec f le nombre de degrés de liberté. A température ambiante $E_t=1/40$ eV donc tout énergie de l'ordre de l'eV est très supérieure à l'énergie thermique.


\paragraph{Chaleur molaire}

Attention ! Si l'on a l'énergie moyenne d'une particule on aura :
\[
	 C=N_a \frac{\partial \bar{E}}{\partial T}
\]

Si on a l'énergie moyenne de l'ensemble:

\[
	 C= \frac{\partial \bar{E}}{\partial T}
\]

\paragraph{Nombre moyen d'occupation}

Dans le cas des fermions (l'explication est analogue à celle pour les bosons voir TD18):
\[
	\bar{N}=\sum_i \frac{g(\epsilon_i)}{e^{-\beta (\epsilon_i-\mu)}+1}
\]


\paragraph{Calculs de densité d'états et états maximaux}

On introduit ici plsusieurs notions importantes qui sur lesquelles ont a pas trop insisté antérieurement.

\begin{itemize}[label=\ding{69}]

	\item Calcul de l'écart énergétique maximal dans un puits 2D : 

	\[
		\Delta E= E(n_{i+1},n_{j})-E(n_{i},n_{j})=\frac{[(n_i+1)^2-n_i^2  ]\pi^2\hbar^2}{2mL^2}= \frac{(2n_i+1) \pi^2 \hbar^2}{2mL^2}\approx \frac{n_i \hbar^2}{mL^2} 
	\]

	Mais comment calculer $n_i$ ? 

	Dans le cas ou tous les niveaux les plus bas d'énergie sont occupés. On aurait par niveau énergétique 2 électrons (en considérant le spin) de plus il faut prendre en compte que le nombre d'états augmente quadratiquement en fonction de $n_i$ ( on peut le voir facilement en calculant les combinaisons possibles de $(n_max=3,3)$. Donc si on assimile au nombre total d'électrons le nombre total d'états différents, une bonne approximation de $n_{max}$ est $\frac{1}{2}\sqrt{\text{Nombre d'électrons}}$. Cependant les états de plus basse énergie ne sont pas remplis la plus part du temps donc un peu faire une approximation de $n_{max}$ en l'augmentant de 2 ordres de grandeur. On verra dans ce TD's que la seule façon qu'ils soient remplis est d'atteindre de très basses températures.


	\item Combien d'états dans une surface S ?
	Si ont attribue à chaque couple $(n_i,n_j)$ un carré de surface 1 u.a Alors le nombre d'états est égal à la surface en question.\\
	Attention aux unités en question.

	\item Comment sont répartis les états en fonction de l'énergie ?
	\[
		n_x^2+n_y^2=\frac{E2mL^2}{\pi^2 \hbar^2}=R^2
	\]

	Mais on ne doit que prendre en compte les $n_i>0$ et le fait que pour chaque état il y a la variation de spin il faudra donc diviser par 4 et multiplier par 2.

\end{itemize} 

\subsection{Résultats}

Les électrons à de basses températures se réunissent dans les niveaux les plus faibles en respectant le principe d'exclusion de Pauli.