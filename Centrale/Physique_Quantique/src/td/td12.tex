\section{TD 12}



\subsection{Rappels}


\paragraph{Flux conductif}

$$\phi=-\lambda \frac{dT}{dx}$$


\paragraph{Équivalence avec l'énergie d'un photon} 

$$E=h\nu=h\frac{c}{\lambda}$$


\paragraph{Quelques valeurs numériques}

\begin{align*}
	h\approx 6,6.10^{-34} && c\approx 3.10^{8} && 1eV\approx 1,6.10^{-19} && 1\text{\AA} = 10^{-10}m
\end{align*}


\paragraph{Le modèle Hydrogenoide} 

\begin{itemize}[label=\ding{69}]

	\item Les couches et leurs occupations 


	\begin{tabular}{|c|c|c|c|}

		\hline
		n & l & Sous-couche électronique & Nombre d'électrons maximum dans la sous-couche \\

		\hline
		n=1 & l=0 & 1s & 2 électrons \\

		\hline
		n=2 & l=0 & 2s & 2 électrons \\

		\hline
		n=2 & l=1 & 2p & 6 électrons \\

		\hline
		n=3 & l=0 & 3s & 2 électrons \\

		\hline
		n=3 & l=1 & 3p & 6 électrons \\

		\hline
		n=3 & l=2 & 3d & 10 électrons \\
		\hline

	\end{tabular}
	 
	\vspace{12pt}

	On pourrait continuer mais le principe ne varie pas on a toujours $2(2l+1)$ possibilités par couche ce qui correspond aux différentes possibilités d'états quantiques. L'impossibilité de cohabitation provient du principe de Pauli.

	\item L'énergie: 

	En valeur absolue
	\[
		E_n= 13,6\cdot\frac{Z^2}{n^2}
	\]

\end{itemize}


\subsection{Notions importantes}


Le couplage L-S, il décrit l'interaction entre les 2 moments dipolaires de l'électron qui orbite autour du noyau. On définit donc 2 nouveaux nombres quantiques à savoir $m_j \ j$ qui vont caractériser le spin total:
\[
	J=L+S
\]

Le couplage L-S va modifier l'énergie des niveaux atomiques on voit apparaitre un autre terme dans le hamiltonien:

\[
	H_{SO}=\xi(r) L\cdot S = \xi(r)\frac{1}{2}(J^2-L^2-S^2)
\]

On verra donc apparaitre l'apparition de 2 nouveaux niveaux d'énergie que l'on notera $2p_{1/2},2p_{3/2}$, qui peuvent accueillir respectivement 2 et 4 électrons. La somme fait bien 6 électrons comme le niveau 2p.

La notion d'intensité d'une raie $K_\alpha$ est couplé à la probabilité de réalisation de la transition. Qui est en fait en directe relation avec le nombre de cas possibles (supérieur pour $2p_{1/2}$ que pour  $2p_{3/2}$) ce qui se connecte avec les principes de la physique statistique.


\subsection{Résultats}


\begin{itemize}[label=\ding{69}]

	\item On a observé que l'on peut identifier un élément de part l'analyse de ses raies de fluorescence qui constituent une signature unique de ce dernier. On peut avec un étalonnage préalable connaitre les proportions de certains éléments sans détruire l'échantillon, ce qui est de grande utilité dans plusieurs domaines (Peintures, alliages..)

	\item L'efficacité de l'effet photo-électrique augmente avec l'intensité et l'énergie du flux de photons incident. Cependant il existe une limité à cette efficacité. 

	\item Il existe deux grandes méthodes d'utilisation des rayons X. La première se base sur la différence d'absorption des rayons X par les différents éléments (radiographie médicale) pour mettre en évidence leur présence. La deuxième utilise la variation des taux d'absorption des différentes espèces pour se centrer sur l'une d'entre elles (Art).

\end{itemize}