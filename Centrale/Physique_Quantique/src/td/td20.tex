\section{TD 20}



\subsection{Notions importantes}


Ce TD n'a pas vraiment de notions importantes autres que l'énergie du puits infini.

\paragraph{vap's vep's puits infini 2D }

\[
	E_{n_x,n_y}=\frac{(n_x^2+n_y^2 )\pi^2\hbar^2}{2mL^2}
\]
\[
	\phi_{n_x,n_y}=Asin(\frac{n_xx\pi}{L})sin(\frac{n_yy\pi}{L})
\]

\paragraph{Le phénomène absorption émission}

Ce phénomène est commun à plusieurs phénomènes physiques que l'on a rencontré au long du cours. Il consiste en l'absorption d'un photon par un électron. On peut voir suite à l'absorption 2 cas. Le premier consiste en l'effet photo-électrique qui conduit à une libération de l'électron de son piège. Dans le deuxième cas l'électron vas passer dans un niveau plus excité. 

Pour ce qui concerne l'émission d'un photon, si on est dans le premier cas elle peut se produire par la volonté d'un électron d'un électron plus excité de prendre la place de celui qui vient d'être éjecté (on se place dans un scénario précis par exemple atome).Dans le deuxième cas l'électron vas se désexciter en émettant un photon.