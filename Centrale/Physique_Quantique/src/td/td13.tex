\section{TD 13}


\subsection{Rappels}


\paragraph{L'équation de Schrodinger}

En régime stationnaire on va résoudre l'équation aux valeurs propres qui nous permet d'obtenir une base sur laquelle on pourra exprimer toutes les fonctions d'état du système.\\
L'équation:
\[
	\hat{H}| \varphi \rangle=E| \varphi \rangle
\]
Il faut de plus prendre en compte les conditions de continuité de la fonction d'onde.

\paragraph{Le puits infini 1D}

Le puits infini est caractérisé par un potentiel de la forme suivante:

\[
	\left\{
	\begin{array}{r c l}
		V(x)=0, & & \quad \forall x \in [0,L] \\
		V(x)=\infty, & & \quad \forall x \notin [0,L]
	\end{array}
	\right.
\]

On aura donc la continuité de la fonction mais pas de la dérivée.


Les conditions de continuité sont:
\[
	\varphi (0)=\varphi (L)=0
\]

On résout l'équation de Schrödinger et on obtiens les résultats suivants:

\begin{align*}
	E_n=\frac{n^2 \pi^2 \hbar^2}{2mL^2} && \langle x|n\rangle= n(x)=sin\left(\frac{n\pi x}{L}\right)
\end{align*}

Ce qui fait apparaitre le résultat très important de partition de l'énergie.


\subsection{Notions importantes}


\paragraph{La 3D}

Il suffit de comprendre que le hamiltonien s'exprime comme la somme de 3 hamiltoniens indépendants:
\[
	\hat{H}=h_x +h_y +h_z
\]
Et si on prend comme fonction d'onde le produit des fonctions d'onde on observe qu'elle est solution de l'équation. On joint la démonstration ci-dessous:
\[
	\hat{H}|\varphi_x \varphi_y \varphi_z\rangle =(h_x +h_y +h_z)|\varphi_x \varphi_y \varphi_z\rangle=
\] 
\[
	h_x|\varphi_x \varphi_y \varphi_z\rangle +h_y|\varphi_x \varphi_y \varphi_z\rangle+h_z|\varphi_x \varphi_y \varphi_z\rangle=E_x|\varphi_x \varphi_y \varphi_z\rangle + E_y|\varphi_x \varphi_y \varphi_z\rangle + E_z|\varphi_x \varphi_y \varphi_z\rangle=
\]
\[
	(E_x +E_y+E_z)|\varphi_x \varphi_y \varphi_z\rangle =E|\varphi_x \varphi_y \varphi_z\rangle
\]

\paragraph{La densité énergétique}

\[
	\frac{dN}{dE}
\]

\paragraph{Approche quantique}

On calcule le nombre d'états entre E et E+dE. On sait qu'ils sont présents sur la sphère.
\[
	n_x^2+n_y^2+n_z^2=\frac{E2mL^2}{\pi^2 \hbar^2}=R^2
\]
Donc calcule le nombre d'états en R et R+dR, l'astuce est d'utiliser si c'est possible l'approximation classique et donc dire que:

\[
	N\approx V
\]


\paragraph{Approche classique}

On utilise la même résolution mais en utilisant l'espace des phases.\\
On sait que:

\begin{align*}
	p=\sqrt{2mE} && dp=\frac{dE\sqrt{2m}}{2\sqrt{E}} && d\Gamma = dr^3dp^3 
\end{align*}

La différence d'approche se base sur l'attribution d'une position a la particule et au lieu de prendre V on prend dR (autour de la particule) que par la suite on confondra.\\
On se rattache au côté quantique en divisant par $\hbar^3$, la justification de cette division est le fait qu'en quantique on a:
\[
	pr>\hbar
\]
ou encore le fait que $p\lambda$ est homogène à $\hbar$