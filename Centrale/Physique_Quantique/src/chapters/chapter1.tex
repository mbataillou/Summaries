\chapter{Les Approximations}



\section{Méthode des perturbations stationnaires}


La théorie des perturbations stationnaires se base sur l'apparition d'un potentiel perturbatif $\hat{W}$ sous la forme d'une quantité arbitrairement petite $\epsilon <<1$ telle que $\hat{W}=\epsilon\hat{V}$ avec V un potentiel fini.\\
En fait il faut valider la condition :
$$\frac{|\langle \varphi_k^{(0)} | \hat{W} |\varphi_i^{(0)} \rangle|}{|E_n^{(0)}-E_p^{(0)}|}<<1$$
Il faut que les perturbations soit petites devant l'écart entre 2 niveaux énergétiques.\\

On trouve les relations suivantes:

\begin{itemize}[label=\ding{69}]

	\item Au premier ordre:

	\begin{itemize}[label=\ding{71}]

		\item L'énergie : 

		\[
			E \approx E_i^{(0)} + \langle \varphi_i^{(0)} | \hat{W} |\varphi_i^{(0)} \rangle
		\]

		\item La fonction propre :

		\[
		 | \varphi \rangle \approx
		| \varphi_i^{(0)} \rangle +
		\sum_{i\neq k}\frac{\langle \varphi_i^{(0)} | \hat{W} |\varphi_k^{(0)} \rangle}{E_i^{(0)}-E_k^{(0)}}
		| \varphi_k^{(0)}\rangle 
		\]

		\item La fonction d'onde :

		\[
		 | \varphi \rangle \approx
		| \varphi_i^{(0)} \rangle +
		\sum_{i\neq k}\frac{|\langle \varphi_i^{(0)} | \hat{W} |\varphi_k^{(0)} \rangle|^2}{E_i^{(0)}-E_k^{(0)}}
		| \varphi_k^{(0)}\rangle
		\] 

	\end{itemize}

	\item Au deuxième ordre:

	\[
		\approx E_i^{(0)} + \langle \varphi_i^{(0)} | \hat{W} |\varphi_i^{(0)} \rangle +\sum_{i\neq k}\frac{|\langle \varphi_k^{(0)} | \hat{W} |\varphi_i^{(0)} \rangle|^2}{E_i^{(0)}-E_k^{(0)}}
	\]

\end{itemize}


\section{La méthode variationelle}


Elle se base sur le fait que si $\phi_t$ est un état quelconque qui s'écrit donc sur la base des états propres $\phi_t=\sum_i c_i |\varphi_i \rangle$ on a :
$$
\langle \phi_t|\hat{H}|\phi_t \rangle \geq E_0 |c_i|^2=E_0\langle \phi_t|\phi_t \rangle$$
On va donc chercher à minimiser la fonctionnelle 
$$E=\frac{\langle \phi_t|\hat{H}|\phi_t \rangle}{\langle \phi_t|\phi_t \rangle}$$
De plus comme $\phi_t$ est une superposition linéaire d'états on utilisera la méthode de Ritz qui dit que minimiser la fonctionnelle revient à résoudre
$$Hc-ESc=0$$
avec la condition que le déterminant $|H-ES|$ soit différent de 0 (pour éviter la solution nulle). Et la normalisation de la fonction d'onde nous donne les solutions de l'approximation.